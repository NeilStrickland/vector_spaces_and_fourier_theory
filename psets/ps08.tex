\begin{document}

\begin{center}
 {\huge Vector Spaces and Fourier Theory ---
   Problem Sheet 8
 }
\end{center}

\begin{rubric}
There will be an online test covering parts of all the questions below.
\end{rubric}

\begin{exercise}\label{ex-innerprod-matrices}
 Use the usual inner product $\ip{A,B}=\trace(AB^T)$ on
 $M_3\R$.  
 \begin{itemize}
  \item[(a)] Calculate all the inner products
   $\ip{C_i,C_j}$, where 
   {\[
    C_1 = \bsm 1&1&1\\0&1&1\\0&0&1 \esm \hspace{3em}
    C_2 = \bsm 1&2&3\\2&1&2\\3&2&1 \esm \hspace{3em}
    C_3 = \bsm 0&1&2\\-1&0&3\\-2&-3&0 \esm
   \]} 
  \item[(b)] Show that if $A^T=A$ and $B^T=-B$ then $A$ and
   $B$ are orthogonal.
  \item[(c)] Put $\vu=[1,1,1]^T$, and let $V$ be the set of all
   matrices $B$ such that the all three columns of $B$ are
   the same.  Show that if $A$ is orthogonal to $V$ then
   $A\vu=0$. 
 \end{itemize}
\end{exercise}
\begin{solution}
 \begin{itemize}
  \item[(a)] 
   \begin{align*}
    \ip{C_1,C_1} &= 1^2+1^2+1^2+0^2+1^2+1^2+0^2+0^2+1^2 = 6 \\
    \ip{C_1,C_2} &= 1.1+1.2+1.3+0.2+1.1+1.2+0.3+0.2+1.1 = 10 \\
    \ip{C_1,C_3} &= 1.0+1.1+1.2+0.(-1)+1.0+1.3+0.(-2)+0.(-3)+1.0 = 6 \\
    \ip{C_2,C_2} &= 1^2+2^2+3^2+2^2+1^2+2^2+3^2+2^2+1^2 = 37 \\
    \ip{C_2,C_3} &= 1.0+2.1+3.2+2.(-1)+1.0+2.3+3.(-2)+2.(-3)+1.0 = 0 \\
    \ip{C_3,C_3} &= 0^2+1^2+2^2+(-1)^2+0^2+3^2+(-2)^2+(-3)^2+0^2 = 28
   \end{align*}
  \item[(b)] Suppose that $A^T=A$ and $B^T=-B$.  We have
   $\ip{A,B}=\trc(AB^T)=-\trc(AB)$.  Using the rules
   $\trc(X)=\trc(X^T)$ and $\trc(YZ)=\trc(ZY)$ we see that
   \[ \trc(AB)=\trc((AB)^T)=\trc(B^TA^T)=
      \trc((-B)A)=-\trc(BA)=-\trc(AB)
   \]
   This means that $\trc(AB)=0$, so $\ip{A,B}=0$.  More
   directly, we have
   {\tiny \[
     A=\bsm a_1&a_2&a_3\\ a_2&a_4&a_5\\ a_3&a_5&a_6\esm
     \hspace{2em}
     B=\bsm 0&b_1&b_2 \\ -b_1&0&b_3 \\ -b_2&-b_3&0\esm
    \]} 
   for some $a_1,\dotsc,a_6,b_1,b_2,b_3$.  It follows that 
   {\tiny \[ AB^T=\bsm
     a_2b_1+a_3b_2 & a_3b_3-a_1b_1 & -a_1b_2-a_2b_3 \\
     a_4b_1+a_5b_2 & a_5b_3-a_2b_1 & -a_2b_2-a_4b_3 \\
     a_5b_1+a_6b_2 & a_6b_3-a_3b_1 & -a_3b_2-a_5b_3
   \esm, \]}
   and the trace of this matrix is zero as required.
  \item[(c)] $V$ is the set of matrices of the form 
   {\tiny \[
     B = \bsm a & a & a \\ b & b & b \\ c & c & c \esm
       = a \bsm 1&1&1\\0&0&0\\0&0&0\esm + 
         b \bsm 0&0&0\\1&1&1\\0&0&0\esm +
         c \bsm 0&0&0\\0&0&0\\1&1&1\esm
   \]}
   Thus, if we put 
   {\tiny \[
       B_1 = \bsm 1&1&1\\0&0&0\\0&0&0\esm \hspace{2em}
       B_2 = \bsm 0&0&0\\1&1&1\\0&0&0\esm \hspace{2em}
       B_3 = \bsm 0&0&0\\0&0&0\\1&1&1\esm
   \]}
   then $V=\spn(B_1,B_2,B_3)$.  Now consider a matrix
   \[ A = \bsm a_1&a_2&a_3\\ a_4&a_5&a_6\\ a_7&a_8&a_9\esm, \]
   and suppose that $A\in V^\perp$.  We then have
   $0=\ip{A,B_1}=a_1+a_2+a_3$ and 
   $0=\ip{A,B_2}=a_4+a_5+a_6$ and
   $0=\ip{A,B_3}=a_7+a_8+a_9$.
   It follows that
   \[ A\bsm 1\\1\\1\esm =
       \bsm a_1+a_2+a_3\\ a_4+a_5+a_6\\ a_7+a_8+a_9\esm
        = \bsm 0\\ 0\\ 0 \esm,
   \]
   as claimed.
 \end{itemize}
\end{solution}

\begin{exercise}
 Use the inner product $\ip{f,g}=\int_{-1}^1 f(x)g(x)\,dx$ on
 $\R[x]_{\leq 2}$.
 \begin{itemize}
  \item[(a)] Find $\ip{x+1,x^2+x}$
  \item[(b)] Show that if $0\leq i,j\leq 2$ and $i+j$ is odd then
   $\ip{x^i,x^j}=0$.
  \item[(c)] Consider a polynomial $u(x)=px^2+q$, and another
   polynomial $f(x)=ax^2+bx+c$.  Give a formula for
   $4f(-1)-8f(0)+4f(1)$ and another formula for $\ip{f,u}$.  Hence
   find $p$ and $q$ such that $\ip{f,u}=4f(-1)-8f(0)+4f(1)$ for all
   quadratic polynomials $f$.
 \end{itemize}
\end{exercise}
\begin{solution}
 \begin{itemize}
  \item[(a)] We have $(x+1)(x^2+x)=x^3+2x^2+x$, so 
   \[ \ip{x+1,x^2+x}=\int_{-1}^1 x^3+2x^2+x\,dx = 
       \left[ \tfrac{1}{4}x^4 + \tfrac{2}{3}x^3 +
        \tfrac{1}{2}x^2 \right]_{-1}^1 = 
       (\tfrac{1}{4}+\tfrac{2}{3}+\tfrac{1}{2}) - 
       (\tfrac{1}{4}-\tfrac{2}{3}+\tfrac{1}{2}) = 4/3.
   \]
  \item[(b)] In general, we have 
   \[ \ip{x^i,x^j} = \int_{-1}^1 x^{i+j}\,dx = 
       \left[ \frac{x^{i+j+1}}{i+j+1} \right]_{-1}^1 = 
        \frac{1}{i+j+1} - \frac{(-1)^{i+j+1}}{i+j+1}.
   \]
   If $i+j$ is odd then $i+j+1$ is even and so $(-1)^{i+j+1}=1$ and
   $\ip{x^i,x^j}=0$.
  \item[(c)] Consider a polynomial $f(x)=ax^2+bx+c$.  We then have 
   \[ 4f(-1)-8f(0)+4f(1) = 4 (a-b+c) -8c + 4(a+b+c) = 8a. \]
   On the other hand, we have
   \begin{align*}
    \ip{f,u} &= \int_{-1}^1 (ax^2+bx+c)(px^2+q)\,dx \\
     &= \int_{-1}^1 apx^4 + bpx^3+ (aq+cp)x^2 + bqx + cq \, dx \\
     &= \left[ \tfrac{ap}{5} x^5 + \tfrac{bp}{4} x^4 + 
         \tfrac{aq+cp}{3} x^3 + \tfrac{bq}{2} x^2 + cqx \right]_{-1}^1 \\
     &= 2\tfrac{ap}{5} + 2\tfrac{aq+cp}{3} + 2cq \\
     &= (\tfrac{2}{5}p+\tfrac{2}{3}q)a + 
        (\tfrac{2}{3}p+2q)c. 
   \end{align*}
   For this to agree with $4f(-1)-8f(0)+4f(1)=8a$, we must have
   $\tfrac{2}{5}p+\tfrac{2}{3}q=8$ and $\tfrac{2}{3}p+2q=0$.  The
   second of these gives $p=-3q$, which we substitute in the first to
   get $-\tfrac{6}{5}q+\tfrac{2}{3}q=8$ and thus $q=-15$.  The
   equation $p=-3q$ now gives $p=45$, so $u(x)=45x^2-15$.
 \end{itemize}
\end{solution}


\begin{exercise}\label{ex-cauchy-i}
 Show that for any $f\in C[-1,1]$ we have 
 \[ \left|\int_{-1}^1\sqrt{1-x^2}\,f(x)\,dx\right| \leq
     \frac{2}{\sqrt{3}} \left(\int_{-1}^1 f(x)^2\,dx\right)^{1/2} 
 \]
 Find a nonzero function $f\in C[-1,1]$ for which the above inequality
 is actually an equality.
\end{exercise}
\begin{solution}
 We use the standard inner product on $C[-1,1]$, given by
 $\ip{f,g}=\int_{-1}^1f(x)g(x)\,dx$.  Take
 $g(x)=\sqrt{1-x^2}$, so 
 \[ \|g\|^2=\int_{-1}^1g(x)^2\,dx=
     \int_{-1}^1 1-x^2\,dx =
     \left[x-\tfrac{1}{3}x^3\right]_{-1}^1 = 4/3,
 \]
 so $\|g\|=2/\sqrt{3}$.  The Cauchy-Schwartz inequality now
 tells us that $|\ip{f,g}|\leq\frac{2}{\sqrt{3}}\|f\|$, or
 in other words 
 \[ \left|\int_{-1}^1\sqrt{1-x^2}\,f(x)\,dx\right| \leq
     \frac{2}{\sqrt{3}} \left(\int_{-1}^1 f(x)^2\,dx\right)^{1/2} 
 \]
 as claimed.  This is an equality iff $f$ is a constant multiple of
 $g$.  In particular, it is an equality when
 $f(x)=g(x)=\sqrt{1-x^2}$. 
\end{solution}

\begin{exercise}\label{ex-cauchy-ii}
 Show that for any $f\in C[0,1]$ we have 
 \[ \left(\int_0^1 f(x)^3\,dx\right)^2 \leq 
     \left(\int_0^1 f(x)^2\,dx\right) 
     \left(\int_0^1 f(x)^4\,dx\right)
 \]
 For which functions $f$ is this actually an equality?
\end{exercise}
\begin{solution}
 We use the standard inner product on $C[0,1]$, given by
 $\ip{f,g}=\int_0^1f(x)g(x)\,dx$.  The Cauchy-Schwartz
 inequality says that for any $f$ and $g$ we have
 $\ip{f,g}^2\leq\|f\|^2\|g\|^2=\ip{f,f}\ip{g,g}$.  Now take
 $g(x)=f(x)^2$, so
 \begin{align*}
  \ip{f,g} &= \int_0^1 f(x)^3\,dx \\
  \ip{f,f} &= \int_0^1 f(x)^2\,dx \\
  \ip{g,g} &= \int_0^1 f(x)^4\,dx.
 \end{align*}
 The inequality therefore says
 \[\left(\int_0^1 f(x)^3\,dx\right)^2 \leq 
     \left(\int_0^1 f(x)^2\,dx\right) 
     \left(\int_0^1 f(x)^4\,dx\right)
 \] 
 as claimed.  This is an equality iff $g$ is a constant multiple of
 $f$, so there is a constant $c$ such that $f^2=cf$, so 
 $(f(x)-c)f(x)=0$.  If $f(x)$ is nonzero for all $x$ we can divide by
 $f(x)$ to see that $f(x)=c$ for all $x$, so $f$ is constant.  The
 same holds by a slightly more complicated argument even if we do not
 assume that $f$ is everywhere nonzero.
\end{solution}

\begin{exercise}\label{ex-innerprod-exotic}
 Consider the space $U=M_2\R$ and the subspace
 $V=\{A\in U\st A^T=A\}$.  Given matrices $A,B\in U$, put
 \[ \ip{A,B} = \det(A-B) - \det(A+B) + 2\trace(A)\trace(B). \]
 Expand out $\ip{A,B}$ when $A=\bsm a_1&a_2\\ a_3&a_4\esm$
 and $B=\bsm b_1&b_2\\ b_3&b_4\esm$.  Show that 
 \begin{itemize}
  \item[(a)] $\ip{A+B,C}=\ip{A,C}+\ip{B,C}$ for all
   $A,B,C\in U$.
  \item[(b)] $\ip{tA,B}=t\ip{A,B}$ for all $A,B\in U$ and
   $t\in\R$.
  \item[(c)] $\ip{A,B}=\ip{B,A}$ for all $A,B\in U$.
  \item[(d)] There exists $A\in U$ such that $\ip{A,A}<0$. 
  \item[(e)] However, if $A\in V$ then $\ip{A,A}\geq 0$,
   with equality iff $A=0$.  
 \end{itemize}
\end{exercise}
\begin{solution}
 If $A=\bsm a_1&a_2\\a_3&a_4\esm$ and
 $B=\bsm b_1&b_2\\ b_3&b_4\esm$ then
 \begin{align*}
  \det(A-B)
    &= \det\bsm a_1-b_1&a_2-b_2\\a_3-b_3&a_4-b_4\esm 
     = (a_1-b_1)(a_4-b_4)-(a_2-b_2)(a_3-b_3) \\
    &= a_1a_4-a_1b_4-a_4b_1+b_1b_4
       -a_2a_3+a_2b_3+a_3b_2-b_2b_3 \\
  \det(A+B)
    &= \det\bsm a_1+b_1&a_2+b_2\\a_3+b_3&a_4+b_4\esm 
     = (a_1+b_1)(a_4+b_4)-(a_2+b_2)(a_3+b_3) \\
    &= a_1a_4+a_1b_4+a_4b_1+b_1b_4
       -a_2a_3-a_2b_3-a_3b_2-b_2b_3 \\
  2\trc(A)\trc(B)
    &= 2(a_1+a_4)(b_1+b_4)
     = 2a_1b_1 + 2a_1b_4 + 2a_4b_1 + 2 a_4b_4 \\
  \ip{A,B} 
    &= -2a_1b_4 -2a_4b_1 + 2a_2b_3 + 2a_3b_2 +
       2a_1b_1 + 2a_1b_4 + 2a_4b_1 + 2 a_4b_4 \\
    &= 2(a_1b_1+a_2b_3+a_3b_2+a_4b_4).
 \end{align*}
 \begin{itemize}
  \item[(a)] We now see that 
   \begin{align*}
    \ip{A+B,C}
     &= 2((a_1+b_1)c_1+(a_2+b_2)c_3+(a_3+b_3)c_2+(a_4+b_4)c_4)\\
     &= 2(a_1c_1+a_2c_3+a_3c_2+a_4c_4) + 
        2(b_1c_1+b_2c_3+b_3c_2+b_4c_4) \\
     &= \ip{A,C} + \ip{B,C}
   \end{align*}
  \item[(b)] Similarly
   \begin{align*}
    \ip{tA,B}
     &= 2(ta_1b_1+ta_2b_3+ta_3b_2+ta_4b_4) \\
     &= t.2(a_1b_1+a_2b_3+a_3b_2+a_4b_4) = t\ip{A,B} 
   \end{align*}
  \item[(c)] It is clear from the formula
   $\ip{A,B}=2(a_1b_1+a_2b_3+a_3b_2+a_4b_4)$ that
   $\ip{A,B}=\ip{B,A}$.
  \item[(d)] In general we have
   $\ip{A,A}=2a_1^2+4a_2a_3+2a_4^2$.  If we take
   $A=\bsm 0&1\\-1&0\esm$ (so $a_1=a_4=0$ and $a_2=1$ and
   $a_3=-1$) then $\ip{A,A}=-4<0$.
  \item[(e)] However, if $A\in V$ then $a_3=a_2$ so
   $\ip{A,A}=2a_1^2+4a_2^2+2a_4^2$.  This is always
   nonnegative, and can only be zero if $a_1=a_2=a_4=0$,
   which means that $A=0$ (because $a_3=a_2$).
 \end{itemize}
\end{solution}

\begin{exercise}\label{ex-innerprod-shm}
 Put
 \[ V = \{f\in C^\infty(\R)\st f+f''=0\}. \]
 For $f,g\in V$ put
 \[ \ip{f,g}(t) = f(t)g(t) + f'(t)g'(t), \]
 so $\ip{f,g}\in C^\infty(\R)$.
 \begin{itemize}
  \item[(a)] Prove that $\ip{f,g}$ is actually a constant.
  \item[(b)] Prove that if $f\in V$ then $f'\in V$, so that
  differentiation gives a linear map $D\:V\to V$.
  \item[(c)] The functions $\sin$ and $\cos$ give a basis for $V$.
   Using this, show that $\ip{,}$ is an inner product on $V$.
  \item[(d)] What is the matrix of $D$ with respect to the basis
  $\{\sin,\cos\}$?
 \end{itemize}
\end{exercise}
\begin{solution}
 \begin{itemize}
  \item[(a)] We have
   \[ \ip{f,g}'=(fg+f'g')'=f'g+fg'+f''g'+f'g''=
       (g+g'')f' + (f+f'')g',
   \]
   and this is zero because $f+f''=0=g+g''$.  Thus $\ip{f,g}'=0$.
  \item[(b)] If $f\in V$ then $f+f''=0$.  Differentiating this
  gives $f'+f'''=0$, which shows that $f'\in V$.
  \item[(c)] It is clear that $\ip{f,g}=\ip{g,f}$ and
   $\ip{f+g,h}=\ip{f,h}+\ip{g,h}$ and $\ip{tf,g}=t\ip{f,g}$.  All
   that is left is to show that $\ip{f,f}\geq 0$, with equality
   only when $f=0$.  For this, we note that
   \begin{align*}
    \ip{\sin,\sin} &= \sin^2 + \cos^2 = 1 \\
    \ip{\cos,\cos} &= \cos^2 + (-\sin)^2 = 1 \\
    \ip{\sin,\cos} &= \sin\cos + \cos.(-\sin) = 0.
   \end{align*}
   Any element $f\in V$ can be written as $f=a.\sin+b.\cos$ for
   some $a,b\in\R$, and we deduce that
   \[ \ip{f,f} =
       a^2\ip{\sin,\sin} + 2ab\ip{\sin,\cos} + b^2\ip{\cos,\cos}
       = a^2 + b^2.
   \]
   From this it is clear that $\ip{f,f}\geq 0$, with equality iff
   $a=b=0$, or equivalently $f=0$.
  \item[(d)]
   We have
   \begin{align*}
    D(\sin) &=  \cos &= 0.\sin + 1.\cos \\
    D(\cos) &= -\sin &= -1.\sin + 0.\cos
   \end{align*}
   It follows that the matrix of $D$ is $\bpm 0&-1\\1&0\epm$.
 \end{itemize}
\end{solution}

\begin{exercise}\label{ex-project-to-symmetric}
 Put $V=\{B\in M_2\R\st B^T=B\}$, and let $\pi\:M_2\R\to V$
 be the orthogonal projection.  Find an orthogonal basis for
 $V$, and use it to calculate $\pi(A)$ for an arbitrary
 matrix $A=\bsm a&b\\ c&d\esm$.  Use this to show that
 $\pi(A)=(A+A^T)/2$. 
\end{exercise}
\begin{solution}
 The obvious basis for $V$ consists of the matrices 
 \[ P_1 = \bsm 1&0\\0&0 \esm \hspace{2em}
    P_2 = \bsm 0&0\\0&1 \esm \hspace{2em}
    P_3 = \bsm 0&1\\1&0 \esm
 \]
 Using the fact that 
 \[ \ip{\bsm a&b \\ c&d\esm, \bsm p&q \\ r&s\esm}
     = ap + bq + cr + ds,
 \]
 we see that the sequence $P_1,P_2,P_3$ is orthogonal, with
 $\ip{P_1,P_1}=\ip{P_2,P_2}=1$ and $\ip{P_3,P_3}=2$.  This
 means that 
 \[ \pi(A) = 
     \ip{A,P_1}P_1 + \ip{A,P_2}P_2 + \half\ip{A,P_3}P_3.
 \]
 Now take $A=\bsm a&b\\ c&d\esm$.  We find that
 $\ip{A,P_1}=a$ and $\ip{A,P_2}=d$ and $\ip{A,P_3}=b+c$, so
 we get 
 \[ \pi(A) = 
     a P_1 + d P_2 + \half(b+c) P_3 = 
     \bsm a & (b+c)/2 \\ (b+c)/2 & d \esm.
 \]
 On the other hand, we have 
 \[ A+A^T = \bsm a & b \\ c & d \esm + 
            \bsm a & c \\ b & d \esm 
          = \bsm 2a & b+c \\ b+c & 2d \esm,
 \]
 so $\pi(A)=(A+A^T)/2$.
\end{solution}

\end{document}



%%% Local Variables:
%%% compile-command: "do_both 08"
%%% End:
