\begin{document}

\begin{center}
 {\huge Vector Spaces and Fourier Theory ---
   Problem Sheet 6
 }
\end{center}

\begin{rubric}
 There will be an online test this week, which covers all the
 questions on this sheet.
\end{rubric}

\begin{exercise}\label{ex-check-commutative}
 Consider the vectors
 \[ \va   = \bsm  3\\  4\\  0\esm \hspace{1em}
    \vb   = \bsm  1\\  2\\  3\esm \hspace{1em}
    \vu_1 = \bsm 16\\-12\\ 15\esm \hspace{1em}
    \vu_2 = \bsm 15\\ 20\\  0\esm \hspace{1em}
    \vu_3 = \bsm 12\\ -9\\-20\esm 
 \] 
 Define $\al\:\R^3\to\R^3$ by $\al(\vx)=\va\tm\vx$.  Let $A$
 be the matrix of $\al$ with respect to the basis
 $\CU=\vu_1,\vu_2,\vu_3$.  
 \begin{itemize}
  \item[(a)] Calculate $\al(\vu_i)$ for $i=1,2,3$, and observe
   that the answer is always a multiple of $\vu_j$ for some
   $j$.
  \item[(b)] Hence write down the matrix $A$. 
  \item[(c)] Calculate $\mu_{\CU}(\vb)$,
   $\al(\mu_\CU(\vb))$, $\phi_A(\vb)$ and
   $\mu_\CU(\phi_A(\vb))$.  Check that
   $\al(\mu_{\CU}(\vb))=\mu_\CU(\phi_A(\vb))$.
 \end{itemize}
\end{exercise}
\begin{solution}
 \begin{itemize}
  \item[(a)] The general formula is
   \[ \al\bsm x\\ y\\ z\esm=\bsm 3\\4\\0\esm\tm\bsm x\\ y\\ z\esm
       = \bsm 4z\\ -3z\\ 3y-4x\esm,
   \]
   so
   \begin{align*}
    \al(\vu_1) &= \bsm 60\\-45\\-100\esm = 5\vu_3 
               &&= 0.\vu_1 + 0.\vu_2 + 5.\vu_3 \\
    \al(\vu_2) &= \bsm 0\\0\\0\esm 
               &&= 0.\vu_1 + 0.\vu_2 + 0.\vu_3 \\
    \al(\vu_3) &= \bsm -80\\60\\-75\esm = -5\vu_1 
               &&= -5\vu_1+0.\vu_2+0.\vu_3
   \end{align*}
  \item[(b)]
   The lists of coefficients here form the columns of the
   matrix $A$, so $A=\bsm 0&0&-5\\ 0&0&0\\ 5&0&0\esm$.
  \item[(c)]
   \begin{align*}
    \mu_\CU(\vb) &= 1.\vu_1 + 2.\vu_2 + 3.\vu_3
                  = \bsm 16\\ -12\\ 15\esm + 
                    \bsm 30\\ 40\\ 0\esm +
                    \bsm 36\\-27\\-60\esm 
                  = \bsm 82\\ 1\\-45\esm \\
    \al(\mu_\CU(\vb)) 
     &= \al\bsm 82\\ 1\\ -45 \esm 
      = \bsm 4\tm (-45) \\ (-3) \tm (-45) \\ 3\tm 1 - 4\tm 82\esm
      = \bsm -180 \\ 135 \\ -325 \esm \\
    \phi_A(\vb) &= A\vb = \bsm 0&0&-5\\ 0&0&0\\ 5&0&0\esm
                          \bsm 1\\ 2\\ 3\esm 
                 = \bsm -15\\ 0\\ 5 \esm \\
    \mu_\CU(\phi_A(\vb)) &= 
     \mu_\CU\bsm -15\\ 0\\ 5\esm =
      -15.\vu_1 + 0.\vu_2 + 5.\vu_3 =
      \bsm (-15)\tm 16 + 5\tm 12 \\
           (-15)\tm(-12) + 5\tm(-9) \\
           (-15)\tm 15 + 5\tm(-20) \esm = 
      \bsm -180 \\ 135 \\ -325 \esm
   \end{align*}
 \end{itemize}
\end{solution}

\begin{exercise}\label{ex-check-composite}
 Define maps $\al,\bt\:M_2\R\to M_2\R$ by 
 \[ \al(X) = X-X^T \hspace{4em}
    \bt(X) = \bsm 1&1\\ 1&1\esm X.
 \]
 Put $\CE=E_1,E_2,E_3,E_4$, where
 \[ E_1 = \bsm 1&0\\0&0\esm \hspace{1em}
    E_2 = \bsm 0&1\\0&0\esm \hspace{1em}
    E_3 = \bsm 0&0\\1&0\esm \hspace{1em}
    E_4 = \bsm 0&0\\0&1\esm. 
 \]
 Let $A$ be the matrix of $\al$ with respect to the basis
 $\CE$, let $B$ be the matrix of $\bt$ with respect to
 $\CE$, and let $C$ be the matrix of $\al\bt$ with respect
 to $\CE$.
 \begin{itemize}
  \item[(a)] Find $\al(E_i)$ for each $i$, and hence find $A$.
  \item[(b)] Find $\bt(E_i)$ for each $i$, and hence find $B$.
  \item[(c)] Find $\al(\bt(E_i))$ for each $i$, and hence
   find $C$.
  \item[(d)] Check that $C=AB$.
 \end{itemize}
\end{exercise}
\begin{solution}
 \begin{itemize}
  \item[(a)] We have $E_1^T=E_1$ and $E_2^T=E_3$ and
   $E_3^T=E_2$ and $E_4^T=E_4$.  It follows that
   $\al(E_1)=\al(E_4)=0$, whereas $\al(E_2)=E_2-E_3$ and
   $\al(E_3)=E_3-E_2$.  From this it follows that
   {\tiny \[
     A = \bsm 0&0&0&0 \\ 0&1&-1&0 \\ 0&-1&1&0\\ 0&0&0&0 \esm.
   \]}
  \item[(b)]
   \begin{align*}
    \bt(E_1) &= \bsm 1&1\\ 1&1\esm \bsm 1&0\\0&0\esm
              = \bsm 1&0\\1&0\esm = 1.E_1+0.E_2+1.E_3 + 0.E_4\\
    \bt(E_2) &= \bsm 1&1\\ 1&1\esm \bsm 0&1\\0&0\esm
              = \bsm 0&1\\0&1\esm = 0.E_1+1.E_2+0.E_3 + 1.E_4\\
    \bt(E_3) &= \bsm 1&1\\ 1&1\esm \bsm 0&0\\1&0\esm
              = \bsm 1&0\\1&0\esm = 1.E_1+0.E_2+1.E_3 + 0.E_4\\
    \bt(E_4) &= \bsm 1&1\\ 1&1\esm \bsm 0&0\\0&1\esm
              = \bsm 0&1\\0&1\esm = 0.E_1+1.E_2+0.E_3 + 1.E_4
   \end{align*}
   The lists of coefficents here give the columns of $B$,
   so
   {\tiny \[
     B = \bsm 1&0&1&0 \\ 0&1&0&1 \\ 1&0&1&0\\ 0&1&0&1 \esm.
   \]}
  \item[(c)] Using part~(b) we get
   \begin{align*}
    \al\bt(E_1) &=\al\bsm 1&0\\1&0\esm 
                 =\bsm 1&0\\1&0\esm -\bsm 1&1\\0&0\esm
                 =\bsm 0&-1\\1&0\esm = 0.E_1-E_2 + E_3 +0.E_4\\
    \al\bt(E_2) &=\al\bsm 0&1\\0&1\esm 
                 =\bsm 0&1\\0&1\esm -\bsm 0&0\\1&1\esm
                 =\bsm 0&1\\-1&0\esm = 0.E_1+E_2 - E_3+0.E_4 \\
    \al\bt(E_3) &=\al\bsm 1&0\\1&0\esm 
                 =\bsm 1&0\\1&0\esm -\bsm 1&1\\0&0\esm
                 =\bsm 0&-1\\1&0\esm = 0.E_1-E_2 + E_3+0.E_4 \\
    \al\bt(E_4) &=\al\bsm 0&1\\0&1\esm 
                 =\bsm 0&1\\0&1\esm -\bsm 0&0\\1&1\esm
                 =\bsm 0&1\\-1&0\esm = 0.E_1+E_2 - E_3+0.E_4
   \end{align*}
   The lists of coefficents here give the columns of $C$,
   so
   {\tiny \[
     C = \bsm 0&0&0&0 \\ -1&1&-1&1 \\ 1&-1&1&-1\\ 0&0&0&0 \esm.
   \]}
  \item[(d)] One checks directly that
   {\tiny \[
     \bsm 0&0&0&0 \\ 0&1&-1&0 \\ 0&-1&1&0\\ 0&0&0&0 \esm
     \bsm 1&0&1&0 \\ 0&1&0&1 \\ 1&0&1&0\\ 0&1&0&1 \esm = 
     \bsm 0&0&0&0 \\ -1&1&-1&1 \\ 1&-1&1&-1\\ 0&0&0&0 \esm,
   \]}
   so $AB=C$.
 \end{itemize}
\end{solution}

\begin{exercise}\label{ex-check-basis-change}
 Let $\al$, $\CE$ and $A$ be as in the previous exercise.
 Now consider the alternative basis
 $\CE'=E'_1,E'_2,E'_3,E'_4$, where 
 \[ E'_1 = \bsm 1&0\\0&1  \esm\hspace{1em}
    E'_2 = \bsm 1&0\\0&-1 \esm\hspace{1em}
    E'_3 = \bsm 0&1\\1&0  \esm\hspace{1em}
    E'_4 = \bsm 0&1\\-1&0 \esm.
 \] 
 Let $P$ be the change of basis matrix from $\CE$ to $\CE'$,
 and let $A'$ be the matrix of $\al$ with respect to $\CE'$.
 \begin{itemize}
  \item[(a)] Express each matrix $E'_i$ as a linear
   combination of $E_1,\dotsc,E_4$, and hence write down the
   matrix $P$.
  \item[(b)] Express each matrix $\al(E'_i)$ as a linear
   combination of $E'_1,\dotsc,E'_4$, and hence write down
   the matrix $A'$.
  \item[(c)] Check that $PA'=AP$.
 \end{itemize}
\end{exercise}
\begin{solution}
 \begin{itemize}
  \item[(a)] The columns of $P$ are the lists of coefficents
   in the following equations:
   \begin{align*}
    E'_1 &= E_1 + 0.E_2 + 0.E_3 + E_4 \\
    E'_2 &= E_1 + 0.E_2 + 0.E_3 - E_4 \\
    E'_3 &= 0.E_1 + E_2 + E_3 + 0.E_4 \\
    E'_4 &= 0.E_1 + E_2 - E_3 + 0.E_4
   \end{align*}
   Thus,
   {\tiny \[ P =
      \bsm 1&1&0&0 \\ 0&0&1&1 \\ 0&0&1&-1 \\ 1&-1&0&0 \esm
   \]}
  \item[(b)] For $i\leq 3$, the matrix $E'_i$ is symmetric,
   so $\al(E'_i)=0$.  This means that the first three
   columns of $A'$ are zero.  We also have
   \[ \al(E'_4) = \bsm 0&1\\-1&0\esm - \bsm 0&-1\\1&0\esm = 
        \bsm 0&2\\-2&0 \esm = 2E'_4
        =0.E'_1+0.E'_2+0.E'_3+2.E'_4,
   \]
   so the last column of $A'$ is $[0,0,0,2]^T$, so 
   {\tiny \[
    A' = \bsm 0&0&0&0 \\ 0&0&0&0 \\ 0&0&0&0 \\ 0&0&0&2 \esm.
   \]}
  \item[(c)] Just by multiplying out we see that
   {\tiny \[
    PA' = 
     \bsm 1&1&0&0 \\ 0&0&1&1 \\ 0&0&1&-1 \\ 1&-1&0&0 \esm
     \bsm 0&0&0&0 \\ 0&0&0&0 \\ 0&0&0&0 \\ 0&0&0&2 \esm =
     \bsm 0&0&0&0 \\ 0&0&0&2 \\ 0&0&0&-2 \\ 0&0&0&0 \esm =
     \bsm 0&0&0&0 \\ 0&1&-1&0 \\ 0&-1&1&0\\ 0&0&0&0 \esm
     \bsm 1&1&0&0 \\ 0&0&1&1 \\ 0&0&1&-1 \\ 1&-1&0&0 \esm = 
    AP
   \]}
 \end{itemize}
\end{solution}

\begin{exercise}\label{ex-char-poly}
 Define $\phi\:\R[x]_{\leq 2}\to\R[x]_{\leq 2}$ by
 $\phi(f)=f+f'+f''$.  Find the matrix of $\phi$ with respect
 to a suitable basis, and hence calculate $\trc(\phi)$,
 $\det(\phi)$ and $\chr(\phi)(t)$.
\end{exercise}
\begin{solution}
 The obvious basis to use is $x^2,x,1$.  We have 
 \begin{align*}
  \phi(x^2) &= x^2 + 2x + 2 \\ 
  \phi(x)  &= 0.x^2 + x + 1 \\
  \phi(1)  &= 0.x^2 + 0.x + 1,
 \end{align*}
 so the matrix of $\phi$ with respect to $x^2,x,1$ is
 \[ P = \bsm 1 & 0 & 0 \\ 2 & 1 & 0 \\ 2 & 1 & 1 \esm \]
 We thus have
 \begin{align*}
  \trc(\phi) &= \trc(P) = 1+1+1 = 3 \\
  \det(\phi) &= \det(P) = 1.\det\bsm 1&0\\1&1\esm  -
                          0.\det\bsm 2&0\\2&1\esm +
                          0.\det\bsm 2&1\\2&1\esm = 1\\
  \chr(\phi)(t) &= \chr(P)(t) = 
    \det\bsm t-1 & 0 & 0 \\ -2 & t-1 & 0 \\ -2 & -1 & t-1 \esm = 
    (t-1)^3.
 \end{align*}
\end{solution}

\begin{exercise}\label{ex-find-jumps}
 Consider the following elements of $\R^6$:
 {\tiny \[
  \vv_1=\bsm 1\\1\\1\\1\\1\\1\esm \hspace{1em}
  \vv_2=\bsm 1\\1\\1\\-1\\-1\\-1\esm \hspace{1em}
  \vv_3=\bsm 0\\0\\0\\1\\1\\1\esm \hspace{1em}
  \vv_4=\bsm 1\\1\\1\\0\\0\\0\esm \hspace{1em}
  \vv_5=\bsm 1\\1\\0\\0\\1\\1\esm \hspace{1em}
  \vv_6=\bsm 0\\0\\1\\1\\0\\0\esm \hspace{1em}
  \vv_7=\bsm 0\\0\\1\\-1\\0\\0\esm \hspace{1em}
  \vv_8=\bsm 1\\1\\0\\0\\-1\\-1\esm \hspace{1em}
 \]}
 Put $\CV=\vv_1,\dotsc,\vv_8$ and $V_0=0$ and 
 $V_j=\spn(\vv_1,\dotsc,\vv_j)$ for $j>0$.
 Recall that $i$ is a \emph{jump} for the sequence $\CV$ if
 $v_i\not\in V_{i-1}$.  Find all the jumps.
\end{exercise}
\begin{solution}
 \begin{enumerate}
  \item Clearly $\vv_1\not\in 0=V_0$, so $1$ is a jump.
  \item Clearly $\vv_2$ is not a multiple of $\vv_1$, so
   $\vv_2\not\in V_1$, so $2$ is a jump.
  \item We have $\vv_3=(\vv_1-\vv_2)/2\in V_2$, so $3$ is
   not a jump.
  \item We have $\vv_4=\vv_1-\vv_3\in V_3$, so $4$ is not a jump.
  \item The vectors $\vv_1,\dotsc,\vv_4$ all have the
   property that the second and third coordinates are the
   same.  Any vector in $V_4=\spn(\vv_1,\dotsc,\vv_4)$ will
   therefore have the same property.  However, the second
   and third coordinates in $\vv_5$ are different, so
   $\vv_5\not\in V_4$, so $5$ is a jump.
  \item We have $\vv_6=\vv_1-\vv_5\in V_5$, so $6$ is not a jump.
  \item The vector $\vv_7$ does not lie in $V_6$.  The
   cleanest way  prove this is to consider the linear map
   $\phi\:\R^6\to\R$ give by
   \[ \phi([x_1,x_2,x_3,x_4,x_5,x_6]^T) =
       x_2-x_3+x_4-x_5.
   \]
   We then find that
   $\phi(\vv_1)=\phi(\vv_2)=\dotsb=\phi(\vv_6)=0$, so
   $\phi(\vu)=0$ for any
   $\vu\in\spn(\vv_1,\dotsc,\vv_6)=V_6$, but
   $\phi(\vv_7)=-2$, so $\vv_7\not\in V_6$.  Thus $7$ is a
   jump.
  \item We have $\vv_8=\vv_2-\vv_7\in V_7$, so $8$ is not a jump.
 \end{enumerate}
 The set of jumps is thus $\{1,2,5,7\}$.
\end{solution}

\end{document}



%%% Local Variables:
%%% compile-command: "do_both 06"
%%% End:
