\begin{document}

\begin{center}
 {\huge Vector Spaces and Fourier Theory ---
   Problem Sheet 5
 }
\end{center}

\begin{rubric}
%Please hand in questions~\ref{ex-matrix-i} and~\ref{ex-independence-proof}
%in the lecture on Monday March 12th.
You are not asked to hand in any questions this week.  
However, solutions will appear on the website next monday as usual,
and you are advised to use them to check your own answers.
\end{rubric}

\begin{exercise}\label{ex-matrix-i}
 Define $\phi\:\R^3\to\R^3$ by 
 \[ \phi\bsm x\\ y\\ z\esm=\bsm y+z\\ z+x\\ x+y\esm. \]
 Find the matrix of $\phi$ with respect to the usual basis of
 $\R^3$.  Then find the matrix with respect to the basis 
 \[ \vu_1 = \bsm  1\\  1\\  1\esm \hspace{2em}
    \vu_2 = \bsm  1\\ -1\\  0\esm \hspace{2em}
    \vu_3 = \bsm  0\\  1\\ -1\esm
 \]
\end{exercise}
\begin{solution}
 We have
 \[ \phi(\ve_1) = \bsm 0\\1\\1\esm \hspace{2em} 
    \phi(\ve_2) = \bsm 1\\0\\1\esm \hspace{2em}
    \phi(\ve_3) = \bsm 1\\1\\0\esm
 \]
 so the matrix with respect to the standard basis is
 \[ \bsm 0 & 1 & 1 \\ 1 & 0 & 1 \\ 1 & 1 & 0 \esm. \]
 On the other hand, we have
 \begin{align*}
   \phi(\vu_1) &= \bsm 2\\2\\2\esm = 2.\vu_1+0.\vu_2+0.\vu_3 \\
   \phi(\vu_2) &= \bsm -1\\ 1\\ 0\esm = 0\vu_1 - 1.\vu_2 + 0.\vu_3 \\
   \phi(\vu_3) &= \bsm 0\\-1\\ 1\esm = 0\vu_1 + 0.\vu_2 - 1.\vu_3
 \end{align*}
 so the matrix with respect to $\vu_1$, $\vu_2$ and $\vu_3$
 is
 \[ \bsm 2 & 0 & 0 \\ 0 & -1 & 0 \\ 0 & 0 & -1 \esm. \]
\end{solution}

\begin{exercise}\label{ex-matrix-ii}
 Define a linear map $\phi\:\R[x]_{\leq 2}\to\R^3$ by
 $\phi(f)=[f(0),f'(1),f''(2)]^T$.  What is the matrix of $\phi$ with
 respect to the usual bases of $\R[x]_{\leq 2}$ and $\R^3$?
\end{exercise}
\begin{solution}
 The usual basis of $\R[x]_{\leq 2}$ is $\{1,x,x^2\}$.  If
 $f(x)=x^2$ then $f'(x)=2x$ and $f''(x)=2$, so $f(0)=0$ and
 $f'(1)=2$ and $f''(2)=2$, so $\phi(f)=[0,2,2]^T$.  In the same
 way, we get
 \[
  \phi(1)   = \bsm 1\\0\\0\esm \hspace{2em}
  \phi(x)   = \bsm 0\\1\\0\esm \hspace{2em}
  \phi(x^2) = \bsm 0\\2\\2\esm.
 \]
 The matrix of $\phi$ has these three vectors as its columns, so
 the matrix is $\bpm 1&0&0\\ 0&1&2\\ 0&0&2\epm$.
\end{solution}

\begin{exercise}\label{ex-matrix-iii}
 Fix a real number $\lm$, and let $V$ be the set of functions of the form
 \[ f(x) = (ax^2 + bx + c)e^{\lm x}. \]
 In other words, we have $V=\R[x]_{\leq 2}e^{\lm x}$.
 \begin{itemize}
  \item[(a)] Write down a basis for $V$.
  \item[(b)] Show that if $f\in V$ then $f'\in V$, so we can
  define a linear map $D\:V\to V$ by $D(f)=f'$.
  \item[(c)] What is the matrix of $D$ with respect to your chosen
  basis?
  \item[(d)] Show that $(D-\lm)^3(f)=0$ for all $f\in V$.
 \end{itemize}
\end{exercise}
\begin{solution}
 \begin{itemize}
  \item[(a)] The functions $f_0(x)=e^{\lm x}$, $f_1(x)=xe^{\lm x}$
  and $f_2(x)=x^2e^{\lm x}$ form a basis for $V$.
  \item[(b)] If $f\in V$ then $f(x)=(ax^2+bx+c)e^{\lm x}$, so
   \[ f'(x)=(2ax+b)e^{\lm x} + (ax^2+bx+c)\lm e^{\lm x}
       = (a\lm x^2+(2a+b\lm)x + (b+c\lm))e^{\lm x}.
   \]
   Here $a$, $b$, $c$ and $\lm$ are all just constants, so we see
   that $f'(x)$ is again a quadratic polynomial times $e^{\lm x}$,
   so $f'\in V$ as required.
  \item[(c)]
   We have
   \begin{align*}
    f'_0 &= \lm f_0 + 0. f_1 + 0.f_2 \\
    f'_1 &= 1.f_0 + \lm f_1 + 0.f_2 \\
    f'_2 &= 0.f_0 + 2.f_1 + \lm f_2
   \end{align*}
   so the matrix is $\bpm \lm&1&0 \\ 0&\lm&2 \\ 0&0&\lm\epm$.
  \item[(d)] We have $(D-\lm)f_0=f'_0-\lm f_0=0$ and similarly
   $(D-\lm)f_1=f_0$ and $(D-\lm)f_2=2f_1$.  It follows that
   $(D-\lm)^2f_2=(D-\lm)f_1=0$ and $(D-\lm)^3f_2=2(D-\lm)^2f_1=0$,
   so $(D-\lm)^3f_i=0$ for $i=0,1,2$, so $(D-\lm)^3=0$.
   Alternatively, we can note that $D-\lm$ has matrix
   $\bpm 0&1&0\\ 0&0&2\\ 0&0&0\epm$, and it is easy to see that
   the cube of this matrix is zero.
 \end{itemize}
\end{solution}

\begin{exercise}\label{ex-complex-eval}
 Put $J=\bpm 0&1\\-1&0\epm\in M_2\R$.  Define
 $\phi\:\R[x]_{\leq 2}\to M_2\R$ by $\phi(f)=f(J)$, or in other
 words
 \[ \phi(ax^2 + bx + c) = aJ^2 + bJ + cI. \]
 Find bases for $\ker(\phi)$ and $\img(\phi)$.
\end{exercise}
\begin{solution}
 We have $J^2=-I$, so if $f(x)=ax^2+bx+c$ we have
 $\phi(f)=(c-a)I+bJ=\bpm c-a&b\\-b&c-a\epm$.  In particular,
 we have $\phi(f)=0$ iff $b=0$ and $c=a$, which means that
 $f(x)=a(x^2+1)$.  We also see that $\img(\phi)$ is spanned
 by $I$ and $J$, which are linearly independent.  Thus
 $\{x^2+1\}$ is a basis for $\ker(\phi)$ and $\{I,J\}$ is a
 basis for $\img(\phi)$.
\end{solution}

\begin{exercise}\label{ex-independence-proof}
 Let $V$ and $W$ be vector spaces, and let $\phi\:V\to W$ be
 a linear map.  Let $\CV=v_1,\dotsc,v_n$ be a list of
 elements of $\CV$. 
 \begin{itemize}
  \item[(a)] Show that if $v_1,\dotsc,v_n$ are linearly
   dependent, then so are $\phi(v_1),\dotsc,\phi(v_n)$. 
  \item[(b)] Give an example where $v_1,\dotsc,v_n$ are
   linearly independent, but $\phi(v_1),\dotsc,\phi(v_n)$
   are linearly dependent. 
  \item[(c)] Show that if $\phi(v_1),\dotsc,\phi(v_n)$ are
   linearly independent, then $v_1,\dotsc,v_n$ are linearly
   independent. 
 \end{itemize}
\end{exercise}
\begin{solution}
 \begin{itemize}
  \item[(a)] If $v_1,\dotsc,v_n$ are linearly dependent,
   then there must exist a linear relation
   $\lm_1v_1+\dotsb+\lm_nv_n=0$ in which one of the
   $\lm_i$'s is nonzero.  We then have
   \[ \lm_1\phi(v_1)+\dotsb+\lm_n\phi(v_n) = 
      \phi(\lm_1v_1+\dotsb+\lm_nv_n) = \phi(0) = 0,
   \]
   which gives a nontrivial linear relation between the
   elements $\phi(v_1),\dotsc,\phi(v_n)$, showing that they
   too are linearly dependent. 
  \item[(b)] Consider $\phi\:\R^2\to\R$ given by
   $\phi\bsm x\\ y\esm=x+y$, and the vectors
   $v_1=\bsm 1\\0\esm$ and $v_2=\bsm 0\\-1\esm$.  Then $v_1$
   and $v_2$ are linearly independent, but
   $\phi(v_1)+\phi(v_2)=0$, which shows that $\phi(v_1)$ and
   $\phi(v_2)$ are linearly dependent.  

   Of course there are many other examples.  The minimal
   example is to let $\phi$ be the map $\R\xra{}0$ given by
   $\phi(t)=0$ for all $t$, and $n=1$, and $v_1=1\in\R$. 
   But this is perhaps so simple as to be confusing. 
  \item[(c)] This is logically equivalent to~(a).  If
   $\phi(v_1),\dotsc,\phi(v_n)$ are linearly independent,
   then $v_1,\dotsc,v_n$ cannot be dependent (as that would
   contradict~(a)), so they must be linearly independent. 
 \end{itemize}
\end{solution}

\begin{exercise}\label{ex-maps-from-quad}
 Let $V$ be a vector space, and let
 $\phi\:\R[x]_{\leq 2}\to V$ be a linear map.  Show that
 there exist elements $u,v\in V$ such that 
 \[ \phi(a x + b) = au + bv \]
 for all $a,b\in\R$. 
\end{exercise}
\begin{solution}
 Put $u=\phi(x)\in V$ and $v=\phi(1)\in V$.  Then
 \[ \phi(ax+b) = \phi(a.x+b.1) = a\phi(x)+b\phi(1)
     = au+bv,
 \]
 as required. 
\end{solution}

\begin{exercise}\label{ex-two-subspaces}
 Define subspaces $V,W\leq\R[x]_{\leq 3}$ by 
 \begin{align*}
  V &= \{f\in\R[x]_{\leq 3}\st f(x)+f(-x)=0\} \\
  W &= \{f\in\R[x]_{\leq 3}\st f''(1)=2f'(1)=6f(1)\}
 \end{align*}
 Find bases for $V$, $W$ and $V\cap W$.  Prove that
 \[ V+W = \{f\in\R[x]_{\leq 3}\st f''(0)=6f(0)\}. \]
\end{exercise}
\begin{solution}
 Firstly, $V$ is just the set of odd polynomials of degree
 at most three.  Any such polynomial has the form $ax^3+cx$
 for some $a,c\in\R$, so $\{x^3,x\}$ is a basis for $V$. 
 Next, consider a polynomial $f(x)=ax^3+bx^2+cx+d$.  We then
 have $f''(1)=6a+2b$ and $f'(1)=3a+2b+c$ and $f(1)=a+b+c+d$,
 so $f\in W$ iff $6a+2b=6a+4b+2c=6a+6b+6c+6d$.  Note that
 $6a+2b=6a+4b+2c$ iff $2b+2c=0$, and $6a+4b+2c=6a+6b+6c+6d$
 iff $2b+4c+6d=0$.  This means that $f\in W$ iff
 $2b+2c=0=2b+4c+6d$, or equivalently $c=-b$ and $d=b/3$.  It
 follows that $W$ is the set of polynomials of the form
 \[ f(x) = ax^3 + b x^2 - b x + b/3 = 
     a x^3 + b(x^2-x+1/3). 
 \]
 This means that $\{x^3,x^2-x+1/3\}$ is a basis for $W$. 
 Next note that a polynomial of the above form can only be
 odd if $b=0$, so $f(x)=ax^3$.  This means that $V\cap W$ is
 the set of polynomials of the form $ax^3$, so $\{x^3\}$ is
 a basis for $V\cap W$. 

 Now put $U=\{f\in\R[x]_{\leq 3}\st f''(0)-6f(0)=0\}$.  If
 $f(x)=ax^3+bx^2+cx+d$ then $f''(0)-6f(0)=2b-6d$, so
 $f\in U$ iff $d=b/3$.  Using this criterion we see that
 $x^3$, $x$  and $x^2-x+1/3$ lie in $U$.  As these three
 polynomials contain a basis for $V$ and also a basis for
 $W$, we see that $V+W\leq U$.  Conversely, suppose we have
 $f(x)$ as above in $U$, so $d=b/3$.  This means that
 \[ f(x) = ax^3 + b(x^2-x+1/3) + (b+c)x \in
      \spn\{x^3,x^2-x+1/3,x\} = V+W. 
 \]
 This shows that $U\leq V+W$, so in fact $U=V+W$ as claimed. 
\end{solution}


\begin{exercise}\label{ex-difference-op}
 Define a map $\Dl\:\R[x]_{\leq 4}\to\R[x]_{\leq 3}$ by
 $\Dl(f(x))=f(x+1)-f(x)$.  What is the matrix of this map with
 respect to the the usual bases of $\R[x]_{\leq 4}$ and
 $\R[x]_{\leq 3}$?  What are the kernel and image of $\Dl$?
\end{exercise}
\begin{solution}
 We have $\Dl(x^k)=(x+1)^k-x^k$, so
 \begin{align*}
  \Dl(x^0) &= 0
           &= 0.x^0 + 0.x^1 + 0.x^2 + 0.x^3 \\
  \Dl(x^1) &= 1
           &= 1.x^0 + 0.x^1 + 0.x^2 + 0.x^3 \\
  \Dl(x^2) &= 2x+1
           &= 1.x^0 + 2.x^1 + 0.x^2 + 0.x^3 \\
  \Dl(x^3) &= 3x^2+3x+1
           &= 1.x^0 + 3.x^1 + 3.x^2 + 0.x^3 \\
  \Dl(x^4) &= 4x^3+6x^2+4x+1
           &= 1.x^0 + 4.x^1 + 6.x^2 + 4.x^3
 \end{align*}
 The matrix is therefore
 \[ \bsm
     0 & 1 & 1 & 1 & 1 \\
     0 & 0 & 2 & 3 & 4 \\
     0 & 0 & 0 & 3 & 6 \\
     0 & 0 & 0 & 0 & 4
    \esm
 \]
 It is clear that the nonzero columns form a basis for
 $\R^4$.  It follows that the map is surjective (so the
 image is all of $\R[x]_{\leq 3}$), and the kernel is just
 the set of constant polynomials.  More explicitly, consider
 a polynomial $p=ax^4+bx^3+cx^2+dx+e$.  We have 
 \begin{align*}
   \Dl(p) &= 
     a(4x^3+6x^2+4x+1)+ b(3x^2+3x+1) + 
     c(2x+1) + d \\
    &= 4ax^3 + (6a+3b)x^2 + (4a+3b+2c)x + (a+b+c+d).
 \end{align*}
 We can only have $\Dl(p)=0$ if
 $4a=6a+3b=4a+3b+2c=a+b+c+d=0$, which is easily solved to
 give $a=b=c=d=0$ (with $e$ arbitrary).  In other words,
 $\Dl(p)$ can only be zero if $p$ is constant.  We also find
 that 
 \begin{align*}
  \Dl(x) &= 1 \\
  \Dl((x^2-x)/2) &= x \\
  \Dl((2x^3-3x^2+x)/6) &= x^2 \\
  \Dl((x^4-2x^3+x^2)/4) &= x^3,
 \end{align*}
 so the image of $\Dl$ is a subspace containing the elements
 $1,x,x^2$ and $x^3$, but these elements span all of
 $\R[x]_{\leq 3}$, so $\Dl$ is surjective.  
\end{solution}

\end{document}



%%% Local Variables:
%%% compile-command: "do_all 05"
%%% End:
