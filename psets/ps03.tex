\begin{document}

\opengraphsfile{pics/pics01}

\begin{center}
 {\huge Vector Spaces and Fourier Theory ---
   Problem Sheet 3
 }
\end{center}

\begin{rubric}
 Please hand in questions 1 and 5 in the lecture on Monday
 3rd March.
\end{rubric}

\begin{exercise}\label{ex-quad-intersection}
 Put $U=\R[x]_{\leq 2}$ and $V=\{f\in U\st f(0)=0\}$ and
 $W=\{f\in U\st f(1)+f(-1)=0\}$.  Show that $V\cap W$ is the
 set of all polynomials of the form $f(x)=bx$, and that
 $V+W=U$.  Please write your argument carefully, using
 complete sentences and correct notation.
\end{exercise}
\begin{solution}
 Consider a polynomial $f\in U$, say $f(x)=ax^2+bx+c$.  We
 have $f(0)=c$, so $f\in V$ iff $c=0$ iff $f(x)=ax^2+bx$ for
 some $a,b\in\R$.  On the other hand, we have 
 \[ f(1) + f(-1) = (a+b+c) + (a-b+c) = 2(a+c), \]
 so $f\in W$ iff $c=-a$, so $f(x)=a(x^2-1)+bx$ for some
 $a,b\in\R$.  Thus $f\in V\cap W$ iff $c=0$ and also $c=-a$,
 which means that $a=c=0$, so $f(x)=bx$ for some $b$.  This
 shows that $V\cap W$ is as claimed.

 On the other hand, given an arbitrary quadratic polynomial
 $f(x)=ax^2+bx+c$ we can put $g(x)=(a+c)x^2$ and
 $h(x)=bx-c(x^2-1)$.  We then have $g(0)=0$ and
 $h(1)+h(-1)=0$, so $g\in V$ and $h\in W$, and $f=g+h$.
 This shows that $V+W=U$.

 \textbf{Aside:} 
 How did we find this $g$ and $h$?  We need $g$ to be an element of
 $V$, so $g$ must have the form $g(x)=px^2+qx$ for some $p,q$.  We
 also need $h$ to be an element of $W$, so $h$ must have the form
 $h(x)=r(x^2-1)+sx$ for some $r,s$.  Finally, we need $f=g+h$, which
 means that 
 \[ ax^2+bx+c = (px^2+qx)+(r(x^2-1)+sx) 
     = (p+r)x^2 + (q+s)x - r.
 \] 
 By comparing coefficients we see that $a=p+r$ and $b=q+s$ and $c=-r$,
 so $r=-c$ and $p=a-r=a+c$ and $s=b-q$ with $q$ arbitrary.  We can
 choose to take $q=0$, giving $s=b$ and so $g(x)=px^2+qx=(a+c)x^2$ and
 $h(x)=r(x^2-1)+sx=-c(x^2-1)+bx$ as before.
\end{solution}

\begin{exercise}\label{ex-inj-misc-i}
 Define $\al\:\R^2\to M_2\R$ by 
 $\displaystyle \al\bsm u\\ v\esm =\bsm u & -u \\ -v & v\esm$.
 Show that $\al$ is injective, and that 
 \[ \img(\al) =\{A\in M_2\R \st A\bsm 1\\1\esm = 0\}. \]
\end{exercise}
\begin{solution}
 Firstly, it is clear that $\al\bsm u\\ v\esm$ can only be
 zero if $u=v=0$, so $\ker(\al)=0$, so $\al$ is injective.
 Next, if $A\in\img(\al)$ then $A=\bsm u& -u\\ v&-v\esm$ for
 some $u$ and $v$, so
 $A\bsm 1\\1\esm=\bsm u&-u\\ v&-v\esm\bsm 1\\1\esm = 
 \bsm 0\\0\esm$.  Conversely, suppose we have a matrix
 $A$ with $A\bsm 1\\1\esm=\bsm 0\\0\esm$.  We can write
 $A=\bsm u&s\\ t&v\esm$ for some $u$, $s$, $t$ and $v$, and
 we must have
 \[ \bsm 0\\0 \esm = \bsm u&s\\ t&v\esm \bsm 1\\1\esm = 
     \bsm u+s\\ t+v\esm.
 \] 
 This means that $s=-u$ and $t=-v$, so
 $A=\bsm u&-u\\ -v&v\esm=\al\bsm u\\ v\esm$, so
 $A\in\img(\al)$.  This proves that
 $\img(\al)=\{A\st A\bsm 1\\1\esm=0\}$, as claimed.
\end{solution}

\begin{exercise}\label{ex-inj-misc-ii}
 Define $\phi\:M_2\R\to M_2\R$ by
 $\phi(A)=A-\half\trace(A)I$.  
 \begin{itemize}
  \item[(a)] Find $\phi\left(\bsm a&b\\ c&d\esm\right)$.
  \item[(b)] Show that $\ker(\phi)=\{aI\st a\in\R\}$.
  \item[(c)] Show that $\img(\phi)=\{A\in M_2\R\st\trace(A)=0\}$.
 \end{itemize}
\end{exercise}
\begin{solution}
 \begin{itemize}
  \item[(a)] $\displaystyle 
   \phi\bsm a&b\\ c&d\esm = 
   \bsm a&b\\ c&d\esm - \frac{a+d}{2}\bsm 1&0\\0&1\esm =
   \bsm (a-d)/2 & b \\ c & (d-a)/2 \esm
   $
  \item[(b)] We have $\trace(I)=2$, so
   $\phi(I)=I-\half.2I=0$, so $aI\in\ker(\phi)$ for all
   $a$.  On the other hand, if $A\in\ker(\phi)$ then
   $A-\half\trace(A)I=0$, so $A=\half\trace(A)I$, which is a
   multiple of $I$.  Alternatively, we can write $A$ as
   $\bsm a&b\\ c&d\esm$, and then if $\phi(A)=0$ then
   part~(a) gives $a-d=b=c=0$, so $A=\bsm a&0\\0&a\esm=aI$.
   Either way, this completes the proof that
   $\ker(\phi)=\{aI\st a\in\R\}$.
  \item[(c)] For any matrix $A=\bsm a&b\\ c&d\esm$, we have
   \[ \trace(\phi(A)) = 
       \trace\left(\bsm (a-d)/2 & b \\ c & (d-a)/2 \esm\right) =
       \frac{a-d}{2} + \frac{d-a}{2} = 0.
   \]
   Thus, $\img(\phi)\sse\{B\in M_2\R\st\trace(B)=0\}$.

   Conversely, suppose we have a matrix $B$ with
   $\trace(B)=0$; we must show that $B\in\img(\phi)$.  We
   thus need to find a matrix $A$ such that $\phi(A)=B$.  In
   fact we have $\phi(B)=B-\half\trace(B)I=B-0.I=B$, so we
   can just take $A=B$.  This completes the proof that
   $\img(\phi)=\{B\st\trace(B)=0\}$. 

   \textbf{Aside:} If we had not noticed that $\phi(B)=B$, what would
   we have done?  We would have $B=\bsm p&q\\ r&-p\esm$ for some
   $p,q,r$, and we would need to find a matrix $A=\bsm a&b\\c&d\esm$
   with $\phi(A)=B$.  Using part~(a) we see that this reduces to the
   equations $(a-d)/2=p$ and $b=q$ and $c=r$ and $(d-a)/2=-p$.  These
   can be solved to give $a=2p+d$ and $b=q$ and $c=r$ with $d$
   arbitrary.  We could take $d=0$, giving $A=\bsm 2p&q\\r&0\esm$, or
   we could take $d=-p$ giving $A=\bsm p&q\\r&-p\esm=B$ as before.
 \end{itemize}
\end{solution}

\begin{exercise}\label{ex-inj-misc-iii}
 Define $\phi\:\R[x]_{\leq 2}\to\R^3$ by 
 \[ \phi(f) =
      \left[ \int_{-1}^0 f(x)\,dx, 
             \int_{-1}^1 f(x)\,dx,
             \int_0^1    f(x)\,dx \right]^T.
 \]
 \begin{itemize}
  \item[(a)] If $f(x)=ax^2+bx+c$, find $\phi(f)$.
  \item[(b)] Show that $\ker(\phi)=\{c(1-3x^2)\st c\in\R\}$.
  \item[(c)] Find a function $g_+(x)=px+q$ such that
   $\phi(g_+)=[1,1,0]^T$.
  \item[(d)] Put $g_-(x)=g_+(-x)$, and show that
   $\phi(g_-)=[0,1,1]^T$. 
  \item[(e)] Deduce that $\img(\phi)=\{[u,v,w]^T\in\R^3\st v=u+w\}$.
 \end{itemize}
\end{exercise}
\begin{solution}
 \begin{itemize}
  \item[(a)] $\displaystyle
    \phi(f) =
     \bbm
      [ax^3/3+bx^2/2+cx]_{-1}^0 \\
      [ax^3/3+bx^2/2+cx]_{-1}^1 \\
      [ax^3/3+bx^2/2+cx]_0^1 
     \ebm = 
     \bbm
      a/3-b/2+c \\ 2a/3+2c \\ a/3+b/2+c
     \ebm
   $
  \item[(b)] We have $f\in\ker(\phi)$ iff $\phi(f)=0$ iff
   $a/3-b/2+c=0$ and $2a/3+2c=0$ and $a/3+b/2+c=0$.  By
   subtracting the first and third equations we see that
   this implies $b=0$, and the second equation gives
   $a=-3c$, so $f(x)=ax^2+bx+c=-3cx^2+c=c(1-3x^2)$.  It
   follows that $\ker(\phi)=\{c(1-3x^2)\st c\in\R\}$.
  \item[(c)] We need 
   \[ \bsm 1\\1\\0\esm = \phi(px+q) = 
      \bsm -p/2+q \\ 2q \\ p/2+q \esm, \]
   so
   \begin{align*}
    -p/2 + q &= 1  \\
    2q       &= 1  \\
     p/2 + q &= 0. 
   \end{align*}
   These equations have the unique solution  $p=-1$ and
   $q=1/2$, so $g_+(x)=\half - x$.  
  \item[(d)] We now have $g_-(x)=\half+x$, and using the
   formula in~(a) we see that $\phi(g_-)=\bsm 0\\1\\1\esm$
   as required.
  \item[(e)] Now put 
   \[ W = \{[u,v,w]^T\in\R^3\st v=u+w\} = 
          \{[u,u+w,w]^T\st u,w\in\R\}.
   \]
   The claim is that $W=\img(\phi)$.  Firstly, for any $f$
   we certainly have
   \[ \int_{-1}^1 f(x)\,dx = 
      \int_{-1}^0 f(x)\,dx +
      \int_0^1 f(x)\,dx,
   \]
   so $\phi(f)\in W$, which proves that $\img(\phi)\sse W$.
   On the other hand, given a vector $\vx=[u,u+w,w]^T\in W$,
   we note that 
   \[ \phi(ug_++wg_-) = 
       u\phi(g_+) + w \phi(g_-) = 
       u \bsm 1\\1\\0\esm + w \bsm 0\\1\\1\esm = 
       \bsm u\\ u+w\\ w\esm = \vx,
   \] 
   so $\vx\in\img(\phi)$.  This shows that
   $W\sse\img(\phi)$, so $W=\img(\phi)$.
 \end{itemize}
\end{solution}

\begin{exercise}\label{ex-inj-misc-iv}
 For each of the following linear maps, decide whether the
 map is injective, whether it is surjective, and whether it
 is an isomorphism.  Please write your arguments carefully,
 using complete sentences and correct notation.  Where
 counterexamples are required, make them as simple and
 specific as possible.
 \begin{itemize}
  \item[(a)] $\phi\:\R^2\to\R^3$ given by
   $\displaystyle \phi\bsm x\\ y\esm=\bsm x\\ y\\ x\esm$
  \item[(b)] $\phi\:\R^3\to\R^2$ given by
   $\displaystyle \phi\bsm x\\ y\\ z\esm=\bsm x-y\\ y-z\esm$
  \item[(c)] $\phi\:\R[x]_{\leq 2}\to\R^3$ given by
   $\phi(f)=[f(0),f'(0),f''(0)]^T$ 
  \item[(d)] $\phi\:\R^2\to M_2\R$ given by 
   $\displaystyle\phi\bsm x\\ y\esm=\bsm 0&x+y\\x+y&0\esm$
  \item[(e)] $\phi\:\R[x]\to\R$ given by $\phi(f)=\int_{-1}^1f(x)\,dx$
 \end{itemize}
\end{exercise}
\begin{solution}
 \begin{itemize}
  \item[(a)] This map is injective but not surjective, and
   so is not an isomorphism.  Indeed, if $\bsm x\\ y\\ x\esm=0$,
   then clearly $\bsm x\\ y\esm=0$.  In other words, if
   $\phi(\vu)=0$, then $\vu=0$, which means that
   $\ker(\phi)=0$, which means that $\phi$ is injective.  On
   the other hand, the vector $\ve_1=\bsm 1\\0\\0\esm$ is
   clearly not of the form $\bsm x\\ y\\ x\esm$, so it does
   not lie in the image of $\phi$, so $\phi$ is not
   surjective. 
  \item[(b)] This map is surjective but not injective, and
   so is not an isomorphism.   Indeed, the vector
   $\vu=\bsm 1\\ 1\\ 1 \esm$ has $\phi(\vu)=0$, so
   $\vu\in\ker(\phi)$, so $\ker(\phi)\neq 0$, so $\phi$ is
   not injective.  On the other hand, for any vector
   $\vv=\bsm p\\ q\esm\in\R^2$ we see that
   \[ \phi\bsm p+q\\q\\0\esm = 
       \bsm (p+q)-q \\ q-0 \esm = \bsm p\\ q\esm = \vv,
   \]
   so $\vv\in\img(\phi)$.  As this works for any
   $\vv\in\R^2$, we deduce that $\img(\phi)=\R^2$, so $\phi$
   is surjective. 

   \textbf{Aside:} How did we find this?  We need to find a vector
   $\vu=\bsm x\\ y\\ z\esm$ with $\phi(\vu)=\vv=\bsm p\\ q\esm$.  This
   reduces to the equations $x-y=p$ and $y-z=q$, giving $x=p+q+z$ and
   $y=q+z$ with $z$ arbitrary.  We could take $z=0$, giving
   $\vu=[p+q,q,0]^T$ as before.  Alternatively, we could take $z=-q$
   giving $\vu=[p,0,-q]^T$.  
  \item[(c)] If $f(x)=ax^2+bx+c$ then 
   \begin{align*}
    f(x)   &= ax^2 + bx + c & f(0)   &= c \\
    f'(x)  &= 2ax + b       & f'(0)  &= b \\
    f''(x) &= 2a            & f''(0) &= 2a,
   \end{align*}
   so 
   \[ \phi(ax^2+bx+c) = [c,b,2a]^T. \]
   Now define $\psi\:\R^3\to\R[x]_{\leq 2}$ by 
   $\psi([p,q,r]^T)=\half r x^2+qx+p$.  We find that
   $\psi(\phi(f))=f$ (for all $f\in\R[x]_{\leq 2}$), and
   $\phi(\psi([p,q,r]^T))=[p,q,r]^T$.  This means that
   $\psi$ is an inverse for $\phi$, so $\phi$ is an
   isomorphism (and so is both injective and surjective). 
  \item[(d)] This is neither injective nor surjective, and
   so is not an isomorphism.  It is not injective because
   $\phi\bsm 1\\-1\esm=0$, which means that $\phi$ has
   nonzero kernel.  Moreover, the matrix $I=\bsm
   1&0\\0&1\esm$ does not have the form
   $\bsm 0&x+y\\ x+y&0\esm$ for any $x$ and $y$, so
   $I\not\in\img(\phi)$, so $\phi$ is not surjective. 
  \item[(e)] This is surjective but not injective, and so is
   not an isomorphism.  It is not injective, because if we
   put $g(x)=x$ then 
   \[ \int_{-1}^1 g(x)\,dx=\int_{-1}^1 x\,dx=
       \left[ \half x^2 \right]_{-1}^1 = 
        \half(1^2 - (-1)^2) = 0. 
   \]
   This shows that $g$ is a nonzero element of $\ker(\phi)$,
   so $\phi$ cannot be injective.  On the other hand, given
   any $t\in\R$ we can let $h(x)$ be the constant function
   with value $t/2$, and we find that
   $\phi(h)=\int_{-1}^1h(x)\,dx=2.(t/2)=t$, showing that
   $t\in\img(\phi)$.  This works for any $t$, so
   $\img(\phi)=\R$, so $\phi$ is surjective. 
 \end{itemize}
\end{solution}

\begin{exercise}\label{ex-inj-misc-v}
 Put 
 \[ L = \{\bsm s \\ 2s \esm \st s\in \R\} \hspace{4em}
    M = \{\bsm 2t \\ t \esm \st t\in \R\}
 \]
 Show that $L\cap M=0$ and $L+M=\R^2$ (or in other words,
 $\R^2=L\op M$). 
\end{exercise}
\begin{solution}
 Consider a vector $\vu$ in $L\cap M$.  We
 must have $\vu=\bsm s\\2s\esm$ for some $s$ (because
 $\vu\in L$) and $\vu=\bsm 2t\\ t\esm$ for some $t$ (because
 $\vu\in M$).  This means that $s=2t$ and $t=2s$.  If we
 substitute the second of these equations in the first we
 get $s=4s$, so $3s=0$, so $s=0$, so $t=0$ and
 $\vu=\bsm 0\\0\esm$.  This shows that $L\cap M=0$. 

 Now consider an arbitrary vector
 $\vu=\bsm x\\ y\esm\in\R^2$.  We want to show that this
 lies in $L+M$, so we must find vectors $\vv\in L$ and
 $\vw\in M$ such that $\vu=\vv+\vw$.  In other words, we
 must find numbers $s$ and $t$ such that
 $\bsm x\\ y\esm=\bsm s\\ 2s\esm + \bsm 2t\\ t\esm$,
 so 
 \begin{align*}
  x &= s+2t \\
  y &= 2s+t. 
 \end{align*}
 These equations have the (unique) solution
 \begin{align*}
  t &= (2x-y)/3 \\
  s &= (2y-x)/3 
 \end{align*}
 so we can put 
 \[ \vv = \bsm s\\ 2s\esm = \bsm (2y-x)/3 \\ (4y-2x)/3 \esm 
    \hspace{4em}
    \vw = \bsm 2t\\ t\esm = \bsm (4x-2y)/3 \\ (2x-y)/3 \esm. 
 \]
 We then have $\vv\in L$ and $\vw\in M$ and $\vu=\vv+\vw$,
 so $\vu\in L+M$.  This works for any vector $\vu\in\R^2$,
 so $\R^2=L+M$ as claimed. 
\end{solution}

\begin{exercise}\label{ex-trace-free-symmetric}
 Put $U=M_3\R$ and $V=\{A\in U\st A^T=A\}$ and
 $W=\{A\in U\st A^T=-A\}$.  Show that $V\cap W=0$ and
 $V+W=U$ (or in other words, $U=V\op W$). 
\end{exercise}
\begin{solution}
 If $A\in V\cap W$ then $A=A^T$ (because $A\in V$) and also
 $A^T=-A$ (because $A\in W$) so $A=-A$.  We now add $A$ to
 both sides to get $2A=0$, and divide by $2$ to get $A=0$. 
 This shows that $V\cap W=0$.  Now consider an arbitrary
 matrix $A\in U$.  Put $A_+=(A+A^T)/2$ and $A_-=(A-A^T)/2$. 
 Then $A_++A_-=A$.  Moreover, we have 
 \begin{align*}
  A_+^T &=(A^T+A^{TT})/2=(A^T+A)/2=(A+A^T)/2 = A_+ \\
  A_-^T &=(A^T-A^{TT})/2=(A^T-A)/2=-(A-A^T)/2 = -A_-
 \end{align*}
 which shows that $A_+\in V$ and $A_-\in W$.  As $A=A_++A_-$
 with $A_+\in V$ and $A_-\in W$, we have $A\in V+W$.  This
 works for any $A\in U$, so $U=V+W$ as claimed. 
\end{solution}

\closegraphsfile
\end{document}



%%% Local Variables:
%%% compile-command: "do_both 03"
%%% End:
