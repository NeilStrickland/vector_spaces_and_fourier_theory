\documentclass[a4paper]{amsart}
%\usepackage{a4wide}
\usepackage{fullpage}
\usepackage[metapost,mplabels,truebbox]{mfpic}
\usepackage[pdftex]{graphicx}
\usepackage[pdftex,debug]{hyperref}
\graphicspath{{pics/}{mfpics/}}
\usepackage{amsmath,amssymb,amsthm}
\usepackage{bm}
\usepackage{verbatim}

\newcommand{\N}         {{\mathbb{N}}}
\newcommand{\Z}         {{\mathbb{Z}}}
\newcommand{\Q}         {{\mathbb{Q}}}
\newcommand{\R}         {{\mathbb{R}}}
\newcommand{\C}         {{\mathbb{C}}}
\newcommand{\bbm}       {\begin{bmatrix}}
\newcommand{\bsm}       {\left[\begin{smallmatrix}}
\newcommand{\ebm}       {\end{bmatrix}}
\newcommand{\esm}       {\end{smallmatrix}\right]}
%\newcommand{\bpm}       {\begin{pmatrix}}
%\newcommand{\epm}       {\end{pmatrix}}
\newcommand{\bpm}       {\left[\begin{matrix}}
\newcommand{\epm}       {\end{matrix}\right]}
\newcommand{\half}      {{\tfrac{1}{2}}}
\newcommand{\qrt}       {{\tfrac{1}{4}}}
\newcommand{\op}        {\oplus}
\newcommand{\hv}        {\hat{v}}
\newcommand{\bcf}[2]{\left(\begin{array}{c}{#1}\\{#2}\end{array}\right)}
\newcommand{\tm}        {\times}
\newcommand{\iffa}      {\Leftrightarrow}
\newcommand{\ip}[1]     {\langle #1\rangle}
\newcommand{\tr}        {\text{trace}}
\newcommand{\ov}        {\overline}

\newcommand{\al}        {\alpha}
\newcommand{\bt}        {\beta}
\newcommand{\gm}        {\gamma}
\newcommand{\dl}        {\delta}
\newcommand{\ep}        {\epsilon}
\newcommand{\zt}        {\zeta}
\newcommand{\et}        {\eta}
\newcommand{\tht}       {\theta}
\newcommand{\io}        {\iota}
\newcommand{\kp}        {\kappa}
\newcommand{\lm}        {\lambda}
\newcommand{\ph}        {\phi}
\newcommand{\ch}        {\chi}
\newcommand{\ps}        {\psi}
\newcommand{\rh}        {\rho}
\newcommand{\sg}        {\sigma}
\newcommand{\om}        {\omega}

\newcommand{\va}        {\mathbf{a}}
\newcommand{\vb}        {\mathbf{b}}
\newcommand{\vc}        {\mathbf{c}}
\newcommand{\vd}        {\mathbf{d}}
\newcommand{\ve}        {\mathbf{e}}
\newcommand{\vf}        {\mathbf{f}}
\newcommand{\vg}        {\mathbf{g}}
\newcommand{\vh}        {\mathbf{h}}
\newcommand{\vi}        {\mathbf{i}}
\newcommand{\vj}        {\mathbf{j}}
\newcommand{\vk}        {\mathbf{k}}
\newcommand{\vl}        {\mathbf{l}}
\newcommand{\vm}        {\mathbf{m}}
\newcommand{\vn}        {\mathbf{n}}
\newcommand{\vo}        {\mathbf{o}}
\newcommand{\vp}        {\mathbf{p}}
\newcommand{\vq}        {\mathbf{q}}
\newcommand{\vr}        {\mathbf{r}}
\newcommand{\vs}        {\mathbf{s}}
\newcommand{\vt}        {\mathbf{t}}
\newcommand{\vu}        {\mathbf{u}}
\newcommand{\vv}        {\mathbf{v}}
\newcommand{\vw}        {\mathbf{w}}
\newcommand{\vx}        {\mathbf{x}}
\newcommand{\vy}        {\mathbf{y}}
\newcommand{\vz}        {\mathbf{z}}

%\newcommand{\vlm}       {\mathbf{\lambda}}
\newcommand{\vlm}       {\bm{\lambda}}

\newcommand{\hw}        {\widehat{w}}

\newcommand{\CA}        {{\mathcal{A}}}
\newcommand{\CB}        {{\mathcal{B}}}
\newcommand{\CC}        {{\mathcal{C}}}
\newcommand{\CD}        {{\mathcal{D}}}
\newcommand{\CE}        {{\mathcal{E}}}
\newcommand{\CP}        {{\mathcal{P}}}
\newcommand{\CQ}        {{\mathcal{Q}}}
\newcommand{\CR}        {{\mathcal{R}}}
\newcommand{\CS}        {{\mathcal{S}}}
\newcommand{\CT}        {{\mathcal{T}}}
\newcommand{\CU}        {{\mathcal{U}}}
\newcommand{\CV}        {{\mathcal{V}}}
\newcommand{\CW}        {{\mathcal{W}}}
\newcommand{\CX}        {{\mathcal{X}}}
\newcommand{\CY}        {{\mathcal{Y}}}


\newcommand{\Dl}        {\Delta}
\newcommand{\sm}        {\setminus}
\newcommand{\sse}       {\subseteq}
\newcommand{\st}        {\;|\;}
\newcommand{\xra}       {\xrightarrow}

\newcommand{\rr}        {\sqrt{3}}
\newcommand{\rs}        {\sqrt{6}}
\newcommand{\rt}        {\sqrt{2}}

\newcommand{\csch}     {\operatorname{csch}}
\newcommand{\sech}     {\operatorname{sech}}
\newcommand{\arcsinh}  {\operatorname{arcsinh}}
\newcommand{\arccosh}  {\operatorname{arccosh}}
\newcommand{\arctanh}  {\operatorname{arctanh}}

\newcommand{\degree}    {\operatorname{degree}}
\newcommand{\range}     {\operatorname{range}}
\newcommand{\trans}     {\operatorname{trans}}
\newcommand{\trc}       {\operatorname{trace}}
\newcommand{\trace}     {\operatorname{trace}}
\newcommand{\chr}       {\operatorname{char}}
\newcommand{\rank}      {\operatorname{rank}}
\newcommand{\adj}       {\operatorname{adj}}
\newcommand{\img}       {\operatorname{image}}
\newcommand{\spn}       {\operatorname{span}}

\newcommand{\VEC}[1]    {\mathbf{#1}}

\newcommand{\BLUEVIOLET}[1]{#1}
\newcommand{\BLUE}[1]{#1}
\newcommand{\BROWN}[1]{#1}
\newcommand{\CORNFLOWERBLUE}[1]{#1}
\newcommand{\DARKORCHID}[1]{#1}
\newcommand{\EMERALD}[1]{#1}
\newcommand{\GREEN}[1]{#1}
\newcommand{\MAGENTA}[1]{#1}
\newcommand{\MAROON}[1]{#1}
\newcommand{\OLIVEGREEN}[1]{#1}
\newcommand{\ORANGE}[1]{#1}
\newcommand{\PINEGREEN}[1]{#1}
\newcommand{\PURPLE}[1]{#1}
\newcommand{\REDVIOLET}[1]{#1}
\newcommand{\RED}[1]{#1}
\newcommand{\ROYALBLUE}[1]{#1}
\newcommand{\RUBINERED}[1]{#1}
\newcommand{\YELLOWORANGE}[1]{#1}


\newcommand{\EMPH}{\emph}
\newcommand{\DEFN}{\emph}
\newcommand{\PMA}{PMA}
\newcommand{\ghost}{}
\newcommand{\mathworld}[1]{}
\newcommand{\exref}[1]{another exercise}
\renewcommand{\:}{\colon}

\renewcommand{\marks}[1]{}
\newcommand{\mks}[1]{}
\newcommand{\mk}{}


\theoremstyle{definition}

\newenvironment{rubric}{}{}
\newtheorem{exercise}{Exercise}
\newenvironment{solution}{{\noindent \bf Solution:}}{}

\begin{document}

\opengraphsfile{pics/pics01}

\begin{center}
 {\huge Vector Spaces and Fourier Theory ---
   Problem Sheet 1
 }
\end{center}

\begin{rubric}
The first two problems on this sheet relate to the week 1
lectures.  The remaining questions are essentially revision
of SOM201 (Linear Algebra for Applications).  Please hand in
questions 2 and 3 in the Monday lecture next week.
\end{rubric}

\begin{exercise}\label{ex-typical-elements}
 For each of the following vector spaces $V$, write down two
 typical elements of $V$ (say $u$ and $v$), calculate $u+v$
 and $10v$, and observe that these are again elements of
 $V$.  (For example, in the case $V=\R^2$ we could take
 $u=\bsm 1\\2\esm$ and $v=\bsm 3\\4\esm$, then we would have
 $u+v=\bsm 4\\6\esm$ and $10v=\bsm 30\\40\esm$, both of
 which are again elements of $\R^2$.)
 \begin{center}
  (a)~~$V=\R^4$  \hspace{3em}
  (b)~~$V=M_{2,3}(\R)$ \hspace{3em}
  (c)~~$V=\R[x]$ \\
  (d) $V$ is the set of physical vectors, as in
   Example~2.6 in the notes.
 \end{center}
\end{exercise}
\begin{solution}
 Of course there are many different correct answers to this
 question. The following will do:
 \begin{itemize}
  \item[(a)] $u=\bsm 1\\1\\1\\1\esm$,
             $v=\bsm 1\\-1\\1\\-1\esm$,
             $u+v=\bsm 2\\0\\2\\0\esm$,
             $10v=\bsm 10\\-10\\10\\-10\esm$.
  \item[(b)] $u=\bsm 1&2&3\\4&5&6\esm$,
             $v=\bsm 6&5&4\\3&2&1\esm$,
             $u+v=\bsm 7&7&7\\7&7&7\esm$,
             $10v=\bsm 60&50&40\\30&20&10\esm$.
  \item[(c)] $u=1+x$, $v=x+x^2$, $u+v=1+2x+x^2$, $10v=10x+10x^2$. 
  \item[(d)] $u$ is the vector pointing 10 miles east, $v$
   is the vector pointing 20 miles west, $u+v$ points ten
   miles west, $10v$ points 200 miles west.
 \end{itemize}
\end{solution}

\begin{exercise}\label{ex-not-vector-spaces}
 Explain why none of the following is a vector space (with
 the obvious definition of addition and scalar
 multiplication). 
 \begin{itemize}
  \item[(a)] $V_0=\left\{\bsm a&b\\ c&d\esm\in M_2\R\st a\leq b\leq
    c\leq d\right\}$
  \item[(b)] $V_1=\left\{\bsm a\\ b\esm\in\R^2\st
                         a+b\text{ is an odd integer } \right\}$
  \item[(c)] $V_2=\left\{\bsm x\\ y\esm\in\R^2 \st x^2=y^2\right\}$.
  \item[(d)] $V_3=\{p\in\R[x]\st p(0)p(1)=0\}$
 \end{itemize}
\end{exercise}
\begin{solution}
 \begin{itemize}
 \item[(a)] This is not a vector space because
  $\bsm 1&2\\3&4\esm\in V_0$ but 
  $(-1).\bsm 1&2\\3&4\esm=\bsm -1&-2\\-3&-4\esm\not\in V_0$,
  which contradicts axiom~(b) of Predefinition~2.1.
 \item[(b)] This is not a vector space because the zero matrix does
  not lie in $V_1$.
 \item[(c)] This is not a vector space because
  $\bsm 1\\1\esm\in V_2$ and $\bsm 1\\-1\esm\in V_2$
  but $\bsm 1\\1\esm+\bsm 1\\-1\esm=\bsm 2\\0\esm\not\in V_2$,
  which contradicts axiom~(a).
 \item[(d)] To see that this is not a vector space, consider
  the polynomials $p(x)=x$ and $q(x)=1-x$.  Then
  $p(0)p(1)=0=q(0)q(1)$, so $p\in V_3$ and $q\in V_3$.
  However, $p(x)+q(x)=1$ for all $x$, so $p+q\not\in V_3$.
  This contradicts axiom~(a).
 \end{itemize}
\end{solution}

\begin{exercise}\label{ex-check-dependence}
 For each of the following lists of vectors, say (with justification) whether
 they are linearly independent, whether they span $\R^3$,
 and whether they form a basis of $\R^3$.  (If you
 understand the concepts involved, you should be able to do
 this by eye, without any calculation.)
 \begin{itemize}
  \item[(a)] $\vu_1=\bsm 1\\0\\2\esm$, 
             $\vu_2=\bsm 3\\0\\4\esm$,
             $\vu_3=\bsm 5\\0\\6\esm$,
             $\vu_4=\bsm 7\\0\\8\esm$.
  \item[(b)] $\vv_1=\bsm 1\\1\\0\esm$,
             $\vv_2=\bsm 1\\0\\1\esm$,
             $\vv_3=\bsm 0\\1\\1\esm$,
             $\vv_4=\bsm 1\\1\\1\esm$.
  \item[(c)] $\vw_1=\bsm 1\\2\\3\esm$,
             $\vw_2=\bsm 4\\5\\6\esm$.
  \item[(d)] $\vx_1=\bsm 1\\1\\1\esm$,
             $\vx_2=\bsm 0\\2\\2\esm$,
             $\vx_3=\bsm 0\\0\\4\esm$.
 \end{itemize}
\end{exercise}
\begin{solution}
 \begin{itemize}
  \item[(a)] These are linearly dependent, and they do not span.  
   Indeed, any list of four vectors in $\R^3$ is always dependent.
   Explicitly, we have $\vu_1-\vu_2-\vu_3+\vu_4=0$, which gives a
   direct proof of dependence.  Also, all the vectors $\vu_i$ have
   zero as the second entry, so the same will be true for any vector
   in the span of the vectors $\vu_i$.  In particular, the vector
   $[0,1,0]^T$ does not lie in that span, so the $\vu_i$'s do not span
   all of $\R^3$.  This means that they do not form a basis.
  \item[(b)] Any list of four vectors in $\R^3$ is
   automatically linearly dependent (and so cannot form a
   basis).  More specifically, the relation
   $\vv_1+\vv_2+\vv_3-2\vv_4=0$ shows that the $\vv_i$'s are
   dependent.  These vectors span all of $\R^3$, because any
   vector $\va=\bsm x\\ y\\ z\esm\in\R^3$ can be expressed
   as $\va=-z\vv_1-y\vv_2-x\vv_3+(x+y+z)\vv_4$.
  \item[(c)] A list of two vectors can only be linearly
   dependent if one is a multiple of the other, which is
   clearly not the case here, so $\vw_1$ and $\vw_2$ are
   linearly independent.  Moreover, a list of two vectors
   can never span all of $\R^3$.  More explicitly, we claim that the
   vector $\ve_1=\bsm 0\\1\\0\esm$ cannot be expressed as a linear
   combination of $\vw_1$ and $\vw_2$.  Indeed, if we have
   $\lm_1\vw_1+\lm_2\vw_2=\ve_1$ then
   \[ \bsm 0\\1\\0\esm =
       \lm_1\bsm 1\\2\\3\esm + \lm_2\bsm 4\\5\\6\esm = 
       \bsm \lm_1+4\lm_2 \\ 2\lm_1+5\lm_2 \\ 3\lm_1 + 6\lm_2 \esm,
   \]
   so 
   \[ \lm_1 + 4\lm_2 = 0 \hspace{4em}
      2\lm_1 + 5\lm_2 = 1 \hspace{4em}
      3\lm_1 + 6\lm_2 = 0.
   \]
   The first and third of these easily give $\lm_1=\lm_2=0$, which is
   incompatible with the second equation, so there is no solution.
   This shows that $\vw_1$ and $\vw_2$ do not form a
   basis of $\R^3$.
  \item[(d)] The vectors $\vx_1$, $\vx_2$ and $\vx_3$ are
   linearly independent and span $\R^3$, so they form a
   basis.  One way to see this is to write down the matrix
   $A=\bsm 1&0&0\\1&2&0\\1&2&4\esm$ whose columns are
   $\vx_1$, $\vx_2$ and $\vx_3$, and observe that it
   row-reduces almost instantly to the identity.
   Alternatively, we must show that for any vector
   $\va=\bsm x\\ y\\ z\esm\in\R^3$, there are unique real
   numbers $\lm,\mu,\nu$ such that 
   \[ \bsm x\\ y\\ z\esm =
       \lm\bsm 1\\1\\1\esm +
       \mu\bsm 0\\2\\2\esm +
       \nu\bsm 0\\0\\4\esm.
   \] 
   This equation is equivalent to $\lm=x$ and
   $\lm+2\mu=y$ and $\lm+2\mu+4\nu=z$.  It is easy to see that there is
   indeed a unique solution, namely $\lm=x$ and $\mu=(y-x)/2$ and
   $\nu=(z-y)/4$. 
 \end{itemize}
\end{solution}

\begin{exercise}\label{ex-subspaces-R-three}
 Consider the following subspaces of $\R^4$:
 \begin{align*}
  U &= \{ [w,x,y,z]^T \st w-x+y-z=0 \} \\
  V &= \{ [w,x,y,z]^T \st w+x+y=0=x+y+z \} \\
  W &= \{ [u,u+v,u+2v,u+3v]^T \st u,v\in\R\}.
 \end{align*}
 Find $U\cap V$, $U\cap W$ and $V\cap W$.
\end{exercise}
\begin{solution}
 \begin{itemize}
  \item[(a)] $U\cap V$ is the set of vectors $[w,x,y,z]^T$
   satisfying the three equations
   \begin{align*}
    w-x+y-z &= 0 \\
    w+x+y &= 0 \\
    x+y+z &= 0.
   \end{align*}
   Subtracting the last two equations gives $w=z$.  Putting
   this back into the first equation gives $x=y$.  The
   middle equation now gives $w=-2x$, so 
   \[ [w,x,y,z] = [-2x,x,x,-2x]. \]
   Thus 
   \[ U\cap V = \{ [-2x,x,x,-2x]^T \st x\in\R\} = 
       \spn([-2,1,1,-2]^T).
   \]
  \item[(b)] $U\cap W$ is the set of vectors of the form
   $[u,u+v,u+2v,u+3v]^T$ for which
   $u-(u+v)+(u+2v)-(u+3v)=0$, which reduces to $-2v=0$, or
   equivalently $v=0$.  Thus 
   \[ U\cap W = \{[u,u,u,u]^T\st u\in\R\} = 
       \spn([1,1,1,1]^T) 
   \]
  \item[(c)] $V\cap W$ is the set of vectors of the form 
   $[u,u+v,u+2v,u+3v]^T$ for which
   $u+(u+v)+(u+2v)=0=(u+v)+(u+2v)+(u+3v)$, or in other words
   $3u+3v=0=3u+6v$.  These equations easily imply that
   $u=v=0$, and this means that $V\cap W=0$.
 \end{itemize}
\end{solution}

\begin{exercise}\label{ex-degenerate-planes}
 Consider the planes $P$, $Q$ and $R$ in $\R^3$ given by 
 \begin{align*}
  P &= \{ [x,y,z]^T \st x+2y+3z=0 \} \\
  Q &= \{ [x,y,z]^T \st 3x+2y+z=0 \} \\
  R &= \{ [x,y,z]^T \st x+y+z=0 \}.
 \end{align*}
 This system of planes has an unusual feature, not shared by
 most other systems of three planes through the origin.
 What is it? 
\end{exercise}
\begin{solution}
 If you choose two planes at random, their intersection will
 be a line (unless the two planes happened to be the same,
 which is unlikely).  If you intersect this with a third
 randomly chosen plane, then you will just get the origin
 (barring unlikely coincidences).  The special feature of
 $P$, $Q$ and $R$ is that $P\cap Q\cap R$ is not just the
 origin, but a line.  Specifically, we have
 \[ P\cap Q\cap R =
      \left\{\bsm t\\ -2t\\ t\esm \st t\in\R\right\}.
 \]
\end{solution}

\begin{exercise}\label{ex-antisym-spectrum}
 Let $a$, $b$ and $c$ be nonzero real numbers.  Show that
 the matrix
 \[ A = \bbm 0 & a & b \\ -a & 0 & c \\ -b & -c & 0 \ebm
 \]
 has one real eigenvalue, and two purely imaginary
 eigenvalues. 
\end{exercise}
\begin{solution}
 The characteristic polynomial is
 \begin{align*}
  \det(tI-A)
   &= \det\bsm t & -a & -b \\
               a & t & -c \\
               b & c & t \esm
    =  t \det\bsm t & -c \\ c & t \esm 
      +a \det\bsm a & -c \\ b & t \esm 
      -b \det\bsm a & t \\ b & c \esm \\
   &= t(t^2+c^2) +a(at+bc) -b(ac-bt) 
    = t^3 + c^2t + a^2t + abc - abc + b^2t \\
   &= t^3+(a^2+b^2+c^2)t.
 \end{align*}
 The eigenvalues of $A$ are the roots of this polynomial.
 These are $0$ (which is real) and $\pm i\sqrt{a^2+b^2+c^2}$
 (which are purely imaginary).
\end{solution}

\begin{exercise}\label{ex-rank}
 Find the rank of the following matrix:
 \[ A = \bsm  1 &  1 &  1 &  1 \\
              1 &  1 & -1 & -1 \\
              1 & -1 &  0 &  0 \\
              0 &  0 &  1 & -1 \esm
 \]
\end{exercise}
\begin{solution}
 The matrix can be row-reduced as follows:
 {\tiny \[
  \bsm  1 &  1 &  1 &  1 \\
        1 &  1 & -1 & -1 \\
        1 & -1 &  0 &  0 \\
        0 &  0 &  1 & -1 \esm \xra{1}
  \bsm  1 &  1 &  1 &  1 \\
        0 &  0 & -2 & -2 \\
        0 & -2 & -1 &  1 \\
        0 &  0 &  1 & -1 \esm \xra{2}
  \bsm  1 &  1 &  0 &  2 \\
        0 &  0 &  0 & -4 \\
        0 & -2 &  0 &  0 \\
        0 &  0 &  1 & -1 \esm \xra{3}
  \bsm  1 &  1 &  0 &  2 \\
        0 &  0 &  0 &  1 \\
        0 &  1 &  0 &  0 \\
        0 &  0 &  1 & -1 \esm \xra{4}
  \bsm  1 &  1 &  0 &  0 \\
        0 &  0 &  0 &  1 \\
        0 &  1 &  0 &  0 \\
        0 &  0 &  1 &  0 \esm \xra{5}
  \bsm  1 &  0 &  0 &  0 \\
        0 &  0 &  0 &  1 \\
        0 &  1 &  0 &  0 \\
        0 &  0 &  1 &  0 \esm \xra{6}
  \bsm  1 &  0 &  0 &  0 \\
        0 &  1 &  0 &  0 \\
        0 &  0 &  1 &  0 \\
        0 &  0 &  0 &  1 \esm.
 \]}
 (In step 1 we subtract row 1 from rows 2 and 3; in step 2
 we subtract suitable multiples of row 4 from rows 1, 2, and
 3; in step 3 we divide rows 2 and 3 by $-4$ and $-2$
 respectively; in steps 4 and 5 we clear the 4th and 2nd
 columns; in step 6 we reorder the rows.)  As the final
 matrix is the identity, we see that $A$ is invertible and
 has rank $4$. 
\end{solution}

\closegraphsfile
\end{document}



%%% Local Variables:
%%% compile-command: "do_both 01"
%%% End:

