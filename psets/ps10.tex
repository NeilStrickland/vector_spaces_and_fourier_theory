\begin{document}

\begin{center}
 {\huge Vector Spaces and Fourier Theory ---
   Problem Sheet 10
 }
\end{center}

\begin{rubric}
 Everywhere on this sheet we use the standard inner products
 on $\R^n$, $M_n\R$ and $T_n$, given by
 $\ip{\vv,\vw}=\sum_{i=1}^nv_iw_i$ and
 $\ip{A,B}=\trc(AB^T)=\trc(A^TB)$ and
 $\ip{f,g}=\frac{1}{2\pi}\int_0^{2\pi}f(t)\overline{g(t)}\,dt$ respectively.

 There will be an online test covering parts of Exercises 1 to 8.
\end{rubric}

\begin{exercise}\label{ex-adjoint-i}
 Consider the map $\phi\:\R^3\to M_2\R$ given by 
 $\phi\bsm x\\ y\\ z\esm = \bsm x& y\\ y&z\esm$.  Given a
 matrix $A=\bsm a&b\\ c&d\esm$, find a vector $\vw=[p,q,r]^T$ such
 that $\ip{\phi(\vv),A}=\ip{\vv,\vw}$ for all vectors
 $\vv\in\R^3$.  (The adjoint map $\phi^*\:M_2\R\to\R^3$ is
 then given by $\phi^*(A)=\vw$.)
\end{exercise}
\begin{solution}
 If $\vv=[x,y,z]^T$ we have 
 \[ \ip{\phi(\vv),A} =
    \ip{\bsm x&y\\ y&z\esm,\bsm a&b\\ c&d\esm} = 
    ax+by+cy+dz = ax + (b+c)y + dz = 
    \ip{\bsm x\\ y\\ z\esm, \bsm a\\ b+c\\ d\esm},
 \]
 so
 \[ \phi^*(A) = \vw = \bsm a\\ b+c\\ d\esm. \]
\end{solution}

\begin{exercise}\label{ex-adjoint-ii}
 Consider the map $\phi\:M_3\R\to M_3\R$ given by 
 \[ \phi\bsm a_1 & a_2 & a_3 \\
             a_4 & a_5 & a_6 \\
             a_7 & a_8 & a_9 \esm = 
        \bsm 0   & a_4 & a_7 \\
             0   & 0   & a_8 \\
             0   & 0   & 0   \esm.
 \]
 Give a formula for the adjoint map
 $\phi^*\:M_3\R\to M_3\R$.
\end{exercise}
\begin{solution}
 We have
 \[ \ip{\bsm 0 & a_4 & a_7 \\ 0 & 0 & a_8 \\ 0 & 0 & 0 \esm,
        \bsm b_1 & b_2 & b_3 \\ b_4 & b_5 & b_6 \\ b_7 & b_8 & b_9 \esm}
    = a_4b_2 + a_7 b_3 + a_8b_6 = 
  \ip{\bsm a_1 & a_2 & a_3 \\ a_4 & a_5 & a_6 \\ a_7 & a_8 & a_9 \esm,
      \bsm 0 & 0 & 0 \\ b_2 & 0 & 0 \\ b_3 & b_6 & 0 \esm}.
 \]
 In other words, if we define
 \[ \psi\bsm b_1 & b_2 & b_3 \\
             b_4 & b_5 & b_6 \\
             b_7 & b_8 & b_9 \esm = 
      \bsm 0 & 0 & 0 \\ b_2 & 0 & 0 \\ b_3 & b_6 & 0 \esm
 \] 
 we have $\ip{\phi(A),B}=\ip{A,\psi(B)}$.  Thus $\phi^*=\psi$. 
\end{solution}

\begin{exercise}\label{ex-adjoint-iii}
 Consider the map $\phi\:M_2\R\to M_2\R$ given by
 $\phi(A)=QAQ$, where $Q=\bsm 1&1\\ 1&1\esm$.  Show that
 $\phi^*=\phi$.  
\end{exercise}
\begin{solution}
 We must show that $\phi^*=\phi$, or equivalently that
 $\ip{\phi(A),B}=\ip{A,\phi(B)}$ for all $A,B\in M_2\R$.
 If $A=\bsm a&b\\ c&d\esm$ and $B=\bsm p&q\\ r&s\esm$ then
 \begin{align*}
  \phi(A) &=
   \bsm 1&1\\ 1&1\esm \bsm a&b\\ c&d\esm\bsm 1&1\\ 1&1\esm 
   = \bsm 1&1\\ 1&1\esm \bsm a+b&a+b \\ c+d\\ c+d\esm 
   = (a+b+c+d) \bsm 1&1\\ 1&1\esm = (a+b+c+d)Q \\
  \ip{Q,B} &= \ip{\bsm 1&1\\ 1&1\esm,\bsm p&q\\ r&s\esm}
            = p+q+r+s \\
  \ip{\phi(A),B} &=
    (a+b+c+d)\ip{Q,B} = (a+b+c+d)(p+q+r+s) \\
  \phi(B) &= (p+q+r+s) Q \\
  \ip{A,\phi(B)} &= (p+q+r+s) \ip{A,Q} = (p+q+r+s)(a+b+c+d) 
    = \ip{\phi(A),B}. 
 \end{align*}
\end{solution}

\begin{exercise}\label{ex-adjoint-iv}
 Define $\phi\:M_n\R\to M_n\R$ by
 $\phi(A)=A-\tfrac{1}{n}\trc(A)I$.  Show that $\phi^*=\phi$.
\end{exercise}
\begin{solution}
 We have
 \begin{align*}
  \ip{\phi(A),B}
   &= \trc(\phi(A)B^T) 
    = \trc(AB^T-\tfrac{1}{n}\trc(A)B^T)
    = \trc(AB^T) - \tfrac{1}{n} \trc(A)\trc(B^T) \\
   &= \ip{A,B} -  \tfrac{1}{n} \trc(A)\trc(B) \\
  \ip{A,\phi(B)} 
   &= \trc(A\phi(B)^T) 
    = \trc(AB^T - \tfrac{1}{n}\trc(B) A) 
    = \trc(AB^T) - \tfrac{1}{n}\trc(A)\trc(B),
 \end{align*}
 so $\ip{\phi(A),B}=\ip{A,\phi(B)}$.  (For a slightly more
 efficient approach, we can note that our expression for
 $\ip{\phi(A),B}$ is symmetric: it does not change if we
 swap $A$ and $B$.  Thus $\ip{\phi(A),B}=\ip{\phi(B),A}$,
 and this is the same as $\ip{A,\phi(B)}$ by the axiom
 $\ip{X,Y}=\ip{Y,X}$.)
\end{solution}

\begin{exercise}\label{ex-adjoint-v}
 In this exercise we give the space $\R[x]_{\leq 2}$ the
 inner product $\ip{f,g}=\int_{-1/2}^{1/2}f(x)g(x)\,dx$.
 Define $\chi\:\R[x]_{\leq 2}\to\R$ by $\chi(f)=f''(0)$.  If
 $f(x)=ax^2+bx+c$, what is $\chi(f)$?  Find an element
 $u\in\R[x]_{\leq 2}$ such that $\chi(f)=\ip{f,u}$ for all
 $f$, and thus give a formula for $\chi^*$.
\end{exercise}
\begin{solution}
 First, if $f(x)=ax^2+bx+c$ then $f'(x)=2ax+b$ and
 $f''(x)=2a$ for all $x$, so in particular
 $\chi(f)=f''(0)=2a$.  This means that $\chi(1)=0$ and
 $\chi(x)=0$ and $\chi(x^2)=2$.

 Next, the element $u$ must have the form $px^2+qx+r$ for some
 constants $p,q,r\in\R$.  It must satisfy
 \begin{align*}
  \ip{1,u} &= \chi(1) = 0 \\
  \ip{x,u} &= \chi(x) = 0 \\
  \ip{x^2,u} &= \chi(x^2) = 2.
 \end{align*}
 On the other hand, we have 
 \begin{align*}
  \ip{1,u}
   &= \int_{-1/2}^{1/2} px^2+qx+r\,dx 
    = \left[ px^3/3 + qx^2/2 + rx \right]_{-1/2}^{1/2}
    = p/12 + r \\
  \ip{x,u}
   &= \int_{-1/2}^{1/2} px^3+qx^2+rx\,dx 
    = \left[ px^4/4 + qx^3/3 + rx^2/2 \right]_{-1/2}^{1/2}
    = q/12 \\
  \ip{x^2,u}
   &= \int_{-1/2}^{1/2} px^4+qx^3+rx^2\,dx 
    = \left[ px^5/5 + qx^4/4 + rx^3/3 \right]_{-1/2}^{1/2}
    = p/80 + r/12
 \end{align*}
 so we must have
 $p/12+r=0$ and $q/12=0$ and $p/80+r/12=2$.  These give
 $p=360$ and $q=0$ and $r=-30$, so $u=360x^2-30$.  

 Now define $\psi\:\R\to\R[x]_{\leq 2}$ by $\psi(t)=tu=360
 tx^2-30t$.  We claim that $\psi$ is adjoint to $\phi$.
 Indeed, the standard inner product on $\R$ is just
 $\ip{s,t}=st$, so
 \[ \ip{\phi(f),t} = t\phi(f)=t\ip{f,u}=\ip{f,tu}=\ip{f,\psi(t)},
 \]
 as required.
\end{solution}

\begin{exercise}
 Let $T_2$ be the usual space of trigonometric polynomials.  We can
 define $\Dl\:T_2\to T_2$ by $\Dl(f)=f''$.
 \begin{itemize}
  \item[(a)] Find $\Dl(f)$, where $f=\sum_{n=-2}^2 a_ne_n$.
  \item[(b)] Show that $\Dl$ is self-adjoint. (This can be deduced
   from part~(a), or you can prove it more directly.)
  \item[(c)] Find the eigenvalues of $\Dl$ (there are three of them).
  \item[(d)] What are the dimensions of the corresponding eigenspaces? 
 \end{itemize}
\end{exercise}
\begin{solution}
 \begin{itemize}
  \item[(a)]
   First note that $e_n(t)=e^{int}$, so $e'_n(t)=ine^{int}$, so
   $e''_n(t)=(in)^2e^{int}=-n^2e_n(t)$.  This means that
   $\Dl(e_n)=-n^2e_n$, and thus that
   $\Dl(f)=\sum_n a_n.(-n^2e_n)=-\sum_n n^2 a_n e_n$.
  \item[(b)]
   Consider elements $f,g\in T_2$, say $f=\sum_{n=-2}^2a_ne_n$ and
   $g=\sum_{k=-2}^2b_me_m$.  We then have 
   \begin{align*}
    \Dl(f) &= -\sum_n n^2 a_n e_n \\   
    \ip{\Dl(f),g} &=
     \left\langle -\sum_n n^2a_ne_n,\sum_m b_me_m\right\rangle 
      = -\sum_{n,m} n^2 a_n\,\overline{b_m}\,\ip{e_n,e_m} \\
    &= -\sum_{n=-2}^2 n^2 a_n \overline{b_n} \\  
    \Dl(g) &= -\sum_m m^2 b_m e_m \\
    \ip{f,\Dl(g)} &=
     \left\langle\sum_n a_ne_n,-\sum_m m^2b_me_m\right\rangle 
      = -\sum_{n,m} m^2 a_n\,\overline{b_m}\,\ip{e_n,e_m} \\
    &= -\sum_{m=-2}^2 m^2 a_m \overline{b_m}
   \end{align*}
   This shows that $\ip{\Dl(f),g}=\ip{f,\Dl(g)}$, so $\Dl$ is
   self-adjoint.
  \item[(c)] Part~(a) tells us that the matrix of $\Dl$ with respect
   to the standard basis $\CE=e_{-2},e_{-1},e_0,e_1,e_2$ is
   {\tiny \[
     D = \bsm -4 &  0 & 0 &  0 &  0 \\
               0 & -1 & 0 &  0 &  0 \\
               0 &  0 & 0 &  0 &  0 \\
               0 &  0 & 0 & -1 &  0 \\
               0 &  0 & 0 &  0 & -4 \esm 
   \]}
   From this it is clear that the eigenvalues are $0$, $-1$ and $-4$.
  \item[(d)] The eigenspace for the eigenvalue $0$ is spanned by $e_0$
   and so has dimension one.  The eigenspace for the eigenvalue $-1$ is
   spanned by $e_{-1}$ and $e_1$, and so has dimension $2$.
   Similarly, the eigenspace for the eigenvalue $-4$ has dimension two,
   with basis $e_{-2},e_2$. 
 \end{itemize}
\end{solution}

\begin{exercise}
 Suppose that $f\in T_2$ satisfies $f(0)=f(\pi)$ and
 $f(-\pi/2)=f(\pi/2)$.  Show that $f(t+\pi)=f(t)$ for all $t$.
\end{exercise}
\begin{solution}
 We can write $f=\sum_{n=-2}^2a_ne_n$ for some sequence of
 coefficients $a_n$.  Note that 
 \begin{align*}
  e_n(0) &= 1 \\
  e_n(\pi) &= e^{in\pi} = (-1)^n \\
  e_n(-\pi/2) &= e^{-in\pi/2} = (e^{-i\pi/2})^n = (-i)^n \\
  e_n(\pi/2) &= e^{in\pi/2} = (e^{i\pi/2})^n = i^n.
 \end{align*}
 It follows that 
 \begin{align*}
  f(0)      &= a_{-2} + a_{-1} + a_0 + a_1 + a_2 \\
  f(\pi)    &= a_{-2} - a_{-1} + a_0 - a_1 + a_2 \\
  f(-\pi/2) &= -a_{-2} + ia_{-1} + a_0 - ia_1 - a_2 \\
  f(\pi/2)  &= -a_{-2} - ia_{-1} + a_0 + ia_1 - a_2 \\
  f(0)-f(\pi) &= 2(a_{-1}+a_1) \\
  f(-\pi/2)-f(\pi/2) &= 2i(a_{-1}-a_1). 
 \end{align*}
 As $f(0)=f(\pi)$ and $f(-\pi/2)=f(\pi/2)$, we must have
 $a_{-1}+a_1=0$ and also $a_{-1}-a_1=0$, which gives $a_{-1}=a_1=0$.
 We therefore have $f=a_{-2}e_{-2}+a_0+a_2e_2$, or in other words
 \[ f(t) = a_{-2} e^{-2it} + a_0 + a_2 e^{2it}. \]
 This gives 
 \[ f(t+\pi) = 
  a_{-2} e^{-2it}e^{-2\pi i} + a_0 + a_2 e^{2it} e^{2\pi i},
 \]
 which is the same as $f(t)$ because $e^{2\pi i}=1$.
\end{solution}

\begin{exercise}
 Put $U=\{f\in T_2\st f(0)=f'(0)=0\}$.  The aim of this exercise is to
 find an orthonormal basis for $U^\perp$.
 \begin{itemize}
  \item[(a)] If $f=\sum_{n=-2}^2a_ne_n$, find $f(0)$ and $f'(0)$ in
   terms of the numbers $a_n$, and so find the general form of an
   element of $U$.
  \item[(b)] Using this, find a basis for $U$.
  \item[(c)] Using this and the fact that $T_2=U\op U^\perp$, find the
   dimensions of $U$ and $U^\perp$.
  \item[(d)] Using part~(a) again, find elements $v_0,v_1\in T_2$ such
   that $\ip{f,v_0}=f(0)$ and $\ip{f,v_1}=f'(0)$ for all $f\in T_2$. 
  \item[(e)] Show that $v_0,v_1$ is an orthogonal basis for $U^\perp$,
   and thus find an orthonormal basis.
 \end{itemize}
\end{exercise}
\begin{solution}
 \begin{itemize}
  \item[(a)] As $f=a_{-2}e_{-2}+a_{-1}e_{-1}+a_0e_0+a_1e_1+a_2e_2$ and
   $e_n(0)=1$ and $e'_n(0)=in$, we have 
   \begin{align*}
    f(0)  &= a_{-2} + a_{-1} + a_0 + a_1 + a_2 \\
    f'(0) &= i.(-2a_{-2} - a_{-1} + a_1 + 2a_2)
   \end{align*}
   This means that $f\in U$ iff we have
   \begin{align*}
    a_{-2} + a_{-1} + a_0 + a_1 + a_2  &= 0 \\
    -2a_{-2} - a_{-1} + a_1 + 2a_2 &= 0
   \end{align*}
   These equations can be solved in a standard way to give
   \begin{align*}
    a_{-2} &= a_0 + 2a_1 + 3 a_2 \\
    a_{-1} &= - 2a_0 - 3a_1 - 4a_2
   \end{align*}
   and so
   \begin{align*}
    f &= (a_0+2a_1+3a_2) e_{-2} - (2a_0+3a_1+4a_2) e_{-1} +
          a_0e_0 + a_1e_1 + a_2e_2 \\
      &= a_0(e_{-2}-2e_{-1}+e_0) + a_1(2e_{-2}-3e_{-1}+e_1) +
         a_2(3e_{-2}-4e_{-1}+e_2).
   \end{align*}
  \item[(b)] From the last expression above, we observe that the
   functions 
   \begin{align*}
    u_0 &= e_{-2}-2e_{-1}+e_0 \\
    u_1 &= 2e_{-2}-3e_{-1}+e_1 \\
    u_2 &= 3e_{-2}-4e_{-1}+e_2
   \end{align*}
   give a basis for $U$.
  \item[(c)] Part~(b) gives a basis for $U$ of length $3$, so
   $\dim(U)=3$.  We also have a basis $e_{-2},e_{-1},e_0,e_1,e_2$ of
   length $5$ for $T_2$, so $\dim(T_2)=5$.  As $T_2=U\op U^\perp$ we have
   $\dim(U)+\dim(U^\perp)=\dim(T_2)=5$, so $\dim(U^\perp)=5-3=2$.
  \item[(d)] Put 
   \begin{align*}
    v_0 &= e_{-2} + e_{-1} + e_0 + e_1 + e_2 \\
    v_1 &= -2e_{-2} -e_{-1} + e_1 + 2e_2.
   \end{align*}
   As the $e_n$'s are orthonormal, we have
   \begin{align*}
    \ip{f,v_0} &= \ip{a_{-2}e_{-2}+a_{-1}e_{-1}+a_0e_0+a_1e_1+a_2e_2,
                      e_{-2} + e_{-1} + e_0 + e_1 + e_2 } \\
               &= a_{-2} + a_{-1} + a_0 + a_1 + a_2 = f(0) \\
    \ip{f,v_1} &= \ip{a_{-2}e_{-2}+a_{-1}e_{-1}+a_0e_0+a_1e_1+a_2e_2,
                      -2e_{-2} - e_{-1} + e_1 + 2 e_2 } \\
               &= -2a_{-2} - a_{-1} + a_1 + 2 a_2 = f'(0).
   \end{align*}
  \item[(e)] If $f\in U$ then $\ip{f,v_0}=f(0)=0$ and
   $\ip{f,v_1}=f'(0)=0$.  This shows that $v_0$ and $v_1$ lie in
   $U^\perp$.  They are clearly linearly independent, as neither one
   is a multiple of the other.  There are two of them, and
   $\dim(U^\perp)=2$ by part~(c), so they give a basis for $U^\perp$.
   Moreover, we have
   \[ \ip{v_0,v_1} = 1.(-1) + 1.(-1) + 1.0 + 1.1 + 1.2 = 0, \]
   so they are orthogonal.  We can therefore construct an orthonormal
   basis of $U^\perp$ by dividing $v_0$ and $v_1$ by their norms.
   Explicitly, we have 
   \begin{align*}
    \|v_0\|^2 &= 1^2 + 1^2 + 1^2 + 1^2 + 1^2 = 5 \\
    \|v_1\|^2 &= (-2)^2 + (-1)^2 + 0^2 + 1^2 + 2^2 = 10
   \end{align*}
   so our orthonormal basis consists of the functions
   \begin{align*}
    \hat{v}_0 &= (e_{-2} + e_{-1} + e_0 + e_1 + e_2)/\sqrt{5} \\
    \hat{v}_1 &= (-2e_{-2} -e_{-1} + e_1 + 2e_2)/\sqrt{10}.
   \end{align*}
 \end{itemize}
\end{solution}

\begin{exercise}\label{ex-isometric-embedding}
 Let $U$ and $V$ be vector spaces with inner products, and
 let $\phi\:U\to V$ be a linear map with the property that
 $\phi^*(\phi(u))=u$ for all $u\in U$.  Let
 $\CU=u_1,\dotsc,u_n$ be an orthonormal sequence in $U$.
 Show that $\phi(u_1),\dotsc,\phi(u_n)$ is an orthonormal
 sequence in $V$.
\end{exercise}
\begin{solution}
 For any $a,b\in U$ we have 
 \[ \ip{\phi(a),\phi(b)}=\ip{a,\phi^*\phi(b)}=\ip{a,b}. \]
 (The first step is the definition of $\phi^*$, and the
 second is the fact that $\phi^*\phi(b)=b$.)  In particular,
 we have $\ip{\phi(u_i),\phi(u_j)}=\ip{u_i,u_j}$.  As the
 sequence $\CU$ is orthonormal, we have $\ip{u_i,u_j}=0$
 when $i\neq j$, and $\ip{u_i,u_i}=1$.  We therefore see
 that $\ip{\phi(u_i),\phi(u_j)}=0$ when $i\neq j$, and
 $\ip{\phi(u_i),\phi(u_i)}=1$, which means that the sequence
 $\phi(u_1),\dotsc,\phi(u_n)$ is also orthonormal.
\end{solution}

\begin{exercise}\label{ex-contraction}
 Let $U$ and $V$ be vector spaces with inner products, and
 let $\phi\:U\to V$ be a linear map with the property that
 $\phi(\phi^*(v))=v$ for all $v\in V$.  Let $u$ be an
 element of $U$, and put $u_1=\phi^*\phi(u)$ and $u_2=u-u_1$.
 \begin{itemize}
  \item[(a)] Show that $\phi(u_1)=\phi(u)$ and $\phi(u_2)=0$. 
  \item[(b)] Show that $\ip{u_1,u_2}=0$.
  \item[(c)] Deduce that $\|u\|^2\geq\|u_1\|^2$
  \item[(d)] Show that $\|u_1\|^2=\|\phi(u)\|^2$.
  \item[(e)] Deduce that $\|\phi(u)\|\leq\|u\|$.
 \end{itemize}
\end{exercise}
\begin{solution}
 \begin{itemize}
  \item[(a)] We are given that $\phi\phi^*(v)=v$ for all
   $v\in V$.  In particular, we can take $v=\phi(u)$ to get
   $\phi\phi^*\phi(u)=\phi(u)$, or in other words
   $\phi(u_1)=\phi(u)$.  It follows that
   $\phi(u_2)=\phi(u-u_1)=\phi(u)-\phi(u_1)=0$. 
  \item[(b)] We have $\ip{\phi^*(a),b}=\ip{a,\phi(b)}$ for
   all $a\in V$ and $b\in U$.  In particular we can take
   $a=\phi(u)$ and $b=u_2$ to get 
   \[ \ip{u_1,u_2} = \ip{\phi^*\phi(u),u_2} = 
      \ip{\phi(u),\phi(u_2)} = 
      \ip{\phi(u),0} = 0.
   \]
   (The penultimate step uses part~(a).)
  \item[(c)] As $u_1$ and $u_2$ are orthogonal we have
   \[ \|u\|^2 = \|u_1+u_2\|^2 = \|u_1\|^2 + \|u_2\|^2 \geq 
       \|u_1\|^2.
   \]
  \item[(d)] We have
   \[ \|u_1\|^2 = \ip{\phi^*\phi(u),\phi^*\phi(u)} = 
       \ip{\phi(u),\phi\phi^*\phi(u)} = \ip{\phi(u),\phi(u)}
       = \|\phi(u)\|^2.
   \]
   (Here we have used the equation
   $\phi\phi^*\phi(u)=\phi(u)$ from part~(a).)
  \item[(e)] If we combine~(c) and~(d) and take square roots
   we get $\|\phi(u)\|\leq\|u\|$.
 \end{itemize}
\end{solution}

\end{document}



%%% Local Variables:
%%% compile-command: "do_both 10"
%%% End:
