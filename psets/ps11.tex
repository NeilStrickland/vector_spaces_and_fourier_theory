\begin{document}

\begin{center}
 {\huge Vector Spaces and Fourier Theory ---
   Problem Sheet 11
 }
\end{center}

\begin{rubric}
This week there is nothing to hand in, and no online test.
\end{rubric}

\begin{exercise}
 Define $\al\:\C^3\to\C^3$ by 
 $\displaystyle\al\bsm x\\ y\\ z\esm =
  \bsm x+2y+z\\ 2x+9y+2z\\ x+2y+z\esm$.
 \begin{itemize}
  \item[(a)] What is the matrix of $\al$ with respect to the standard
   basis of $\C^3$?
  \item[(b)] Show that $\al$ is self-adjoint.
  \item[(c)] Find an orthonormal basis of $\C^3$ consisting of
   eigenvectors of $\al$.
 \end{itemize}
\end{exercise}
\begin{solution}
 \begin{itemize}
  \item[(a)] We have $\al(\vx)=A\vx$, where 
   $\displaystyle A=\bsm 1&2&1\\ 2&9&2\\ 1&2&1\esm$, so $A$ is the
   matrix that we need.
  \item[(b)] We have $\al=\phi_A$ and so $\al^\dag=\phi_{A^\dag}$, but
   clearly $A^\dag=A$, so $\al$ is self-adjoint.
  \item[(c)] Here we just need to find the eigenvalues and
   eigenvectors of the matrix $A$.  We have
   \begin{align*}
    \det(A-tI) &=
       \det\bsm 1-t & 2 & 1 \\ 2 & 9-t & 2 \\ 1 & 2 & 1-t \esm 
     = (1-t)\det\bsm 9-t & 2 \\ 2 & 1-t \esm 
         -2 \det\bsm 2 & 2 \\ 1 & 1-t \esm 
          + \det\bsm 2 & 9-t \\ 1 & 2 \esm \\
    &= (1-t)(t^2-10t+5)  - 2(-2t) + (t-5) 
     = -t^3 + 11t^2 - 10t  \\
    &= -t(t^2-11t+10) = -t(t-1)(t-10),
   \end{align*}
   so the characteristic polynomial is $\det(tI-A)=t(t-1)(t-10)$, so
   the eigenvalues are $0$, $1$ and $10$.  For an eigenvector of
   eigenvalue $0$ we must have 
   \begin{align*}
    x+2y+z &= 0 \\
    2x+9y+2z &= 0 \\
    x+2y+z &= 0.
   \end{align*}
   These equations give $y=0$ and $z=-x$, so any eigenvector of
   eigenvalue $0$ is a multiple of $[1,0,-1]^T$.  For a unit vector,
   we can take $u_1=[1,0,-1]^T/\sqrt{2}$.

   For an eigenvector of eigenvalue $1$ we must have 
   \begin{align*}
    x+2y+z &= x \\
    2x+9y+2z &= y \\
    x+2y+z &= z.
   \end{align*}
   These equations give $x=z=-2y$, so any eigenvector of eigenvalue
   $1$ is a multiple of $[-2,1,-2]^T$.  For a unit vector, we can take
   $u_2=[-2,1,-2]^T/3$.    

   For an eigenvector of eigenvalue $10$ we must have 
   \begin{align*}
    x+2y+z &= 10x \\
    2x+9y+2z &= 10y \\
    x+2y+z &= 10z.
   \end{align*}
   These equations give $z=x$ and $y=4x$, so any eigenvector of eigenvalue
   $1$ is a multiple of $[1,4,1]^T$.  For a unit vector, we can take
   $u_3=[1,4,1]^T/(3\sqrt{2})$.

   Now $u_1$, $u_2$ and $u_3$ are eigenvectors of a self-adjoint
   map with different eigenvalues, so they are automatically
   orthogonal.  (It is easy to check this directly, of course.)  They
   are unit vectors, so they give an orthonormal (and so linearly
   independent) sequence.  As this is an independent sequence of
   length three in a three-dimensional space, it must be a basis.
 \end{itemize}
\end{solution}

\begin{exercise}
 Define $\al\:\C^5\to\C^5$ by 
 \[ \al([z_0,z_1,z_2,z_3,z_4]^T) = [z_1,z_2,z_3,z_4,z_0]^T. \]
 \begin{itemize}
  \item[(a)] Find $\al^\dag$, and show that $\al^\dag=\al^{-1}$.
   (Of course this is a special property of this particular map.  For
   most linear maps, the adjoint is unrelated to the inverse.)
  \item[(b)] Find the eigenvalues of $\al$.  (Hint: it is easier to
   think directly about when $\al(z)=\lm\,z$, rather than trying to
   calculate the characteristic polynomial.  You will need to consider
   separately the cases where $\lm^5=1$ and where $\lm^5\neq 1$.)
 \end{itemize}
\end{exercise}
\begin{solution}
 \begin{itemize}
  \item[(a)] If $z=[z_0,\dotsc,z_4]^T$ and $w=[w_0,\dotsc,w_4]^T$ we
   have 
   \begin{align*}
    \ip{\al(z),w} &= 
     \ip{[z_1,z_2,z_3,z_4,z_0]^T,[w_0,w_1,w_2,w_3,w_4]^T} \\
    &= z_1\ov{w_0} + z_2\ov{w_1} + z_3\ov{w_2} + z_4\ov{w_3} + z_0\ov{w_4} \\
    &= z_0\ov{w_4} + z_1\ov{w_0} + z_2\ov{w_1} + z_3\ov{w_2} + z_4\ov{w_3} \\
    &= \ip{[z_0,z_1,z_2,z_3,z_4]^T,[w_4,w_0,w_1,w_2,w_3]^T}. \\
   \end{align*}
   Thus, if we define $\bt\:\C^5\to\C^5$ by 
   \[ \bt([w_0,w_1,w_2,w_3,w_4]^T) = [w_4,w_0,w_1,w_2,w_3]^T, \]
   we have $\ip{\al(z),w}=\ip{z,\bt(w)}$.  This means that
   $\bt=\al^\dag$.  We also have
   \begin{align*}
    \bt\al(z) &= \bt([z_1,z_2,z_3,z_4,z_0]^T) = z \\
    \al\bt(w) &= \al([w_4,w_0,w_1,w_2,w_3]^T) = w,
   \end{align*}
   so $\bt$ is inverse to $\al$.  Thus $\al^{-1}=\bt=\al^\dag$.
  \item[(b)] Suppose that $\al(z)=\lm\,z$, or in other words 
   \[ [z_1,z_2,z_3,z_4,z_0] =
       [\lm z_0,\lm z_1,\lm z_2,\lm z_3,\lm z_4].
   \]
   This means that
   \begin{align*}
    z_1 &= \lm z_0 \\
    z_2 &= \lm z_1 = \lm^2 z_0 \\
    z_3 &= \lm z_2 = \lm^3 z_0 \\
    z_4 &= \lm z_3 = \lm^4 z_0 \\
    z_0 &= \lm z_4 = \lm^5 z_0.
   \end{align*}
   The last equation gives $(\lm^5-1)z_0=0$.  If $\lm^5\neq 1$ then we
   see that $z_0=0$, and by substituting this into our other
   equations we see that $z_1=z_2=z_3=z_4=0$ as well, so $z=0$.  Thus,
   for such $\lm$ there are no nonzero eigenvectors, so $\lm$ is not
   an eigenvalue.  

   Suppose instead that $\lm^5=1$, which means that $\lm=e^{2\pi
    ik/5}$ for some $k\in\{0,1,2,3,4\}$.  We then see that the vector 
   \[ u_k = [1,\lm,\lm^2,\lm^3,\lm^4]^T \]
   is an eigenvector of eigenvalue $\lm$.  Thus, the eigenvalues are
   precisely the fifth roots of unity.
 \end{itemize}
\end{solution}

\begin{exercise}
 Define a map $\al\:\R[x]_{\leq 2}\to\R[x]_{\leq 2}$ by
 $\al(f)=(3x^2-1)f''$.  You may assume that if we use the inner
 product $\ip{f,g}=\int_{-1}^1 f(x)g(x)\,dx$ on $\R[x]_{\leq 2}$, then
 $\al$ is self-adjoint.
 \begin{itemize}
  \item[(a)] Show that $\al(\al(f))=6\al(f)$ for all $f$.
  \item[(b)] Deduce that if $f$ is a nonzero eigenvector of $\al$ with
   eigenvalue $\lm$, then $\al(\al(f))=\lm^2f$ and $\lm^2=6\lm$.
  \item[(c)] Find an orthogonal basis for $\R[x]_{\leq 2}$ consisting
   of eigenvectors for $\al$.
 \end{itemize}
\end{exercise}
\begin{solution}
 \begin{itemize}
  \item[(a)] If $f=ax^2+bx+c$ then $f''=2a$, so $\al(f)=6ax^2-2a$, so
   $\al(f)''=12a$, so
   $\al(\al(f))=(3x^2-1)\al(f)''=36ax^2-12a=6\al(f)$.
  \item[(b)] If $\al(f)=\lm f$ then
   $\al(\al(f))=\al(\lm f)=\lm\al(f)=\lm.\lm f=\lm^2 f$.  On the other
   hand, we also know from~(a) that $\al(\al(f))=6\al(f)=6\lm f$, so
   $\lm^2 f=6\lm f$, so $(\lm^2-6\lm)f=0$.  As $f$ was assumed to be
   nonzero, this means that $\lm^2-6\lm=0$, or $\lm(\lm-6)=0$, so
   $\lm=0$ or $\lm=6$.
  \item[(c)] By part~(b) the only possible eigenvalues are $0$ and
   $6$.  Clearly $\al(f)=0$ iff $f''=0$ iff $a=0$, so the eigenvectors
   of eigenvalue $0$ are just the polynomials of the form $bx+c$.
   In particular, if we put $u_1=1$ and $u_2=x$ then $u_1$ and $u_2$
   are eigenvectors, and we have $\ip{u_1,u_2}=\int_{-1}^1 x\,dx=0$,
   so they are orthogonal.  Next take $u_2=3x^2-1$.  By definition
   we have $\al(f)=u_2.f''$ for all $f$, so in particular
   $\al(u_2)=u_2.u_2''=6u_2$, so $u_2$ is an eigenvector of eigenvalue
   $6$.  We are given that $\al$ is self-adjoint, so eigenvectors of
   different eigenvalues are automatically orthogonal, so
   $u_1,u_2,u_3$ is an orthogonal sequence.  
 \end{itemize}
\end{solution}

\begin{exercise}
 Let $V$ be the space of functions $f\:\R\to\C$ of the form
 $f(x)=p(x)e^{-x^2/2}$ for some $p\in\C[x]$.  Give this the Hermitian
 form $\ip{f,g}=\int_{-\infty}^\infty f(x)\ov{g(x)}\,dx$.  Define
 $\phi\:V\to V$ by $\phi(f)(x)=x\,f(x)$.
 \begin{itemize}
  \item[(a)] Show that $\phi$ is self-adjoint.
  \item[(b)] Show that if $\phi(f)=\lm f$ for some $\lm\in\C$, then
   $f=0$.  (Thus, $\phi$ has no eigenvalues.  This is only possible
   because $V$ is infinite-dimensional.)
 \end{itemize}
\end{exercise}
\begin{solution}
 \begin{itemize}
  \item[(a)] We have
   $\ip{\phi(f),g}=\int_{-\infty}^\infty x\,f(x)\ov{g(x)}\,dx$ and
   $\ip{f,\phi(g)}=\int_{-\infty}^\infty f(x)\ov{x\,g(x)}\,dx=
   \int_{-\infty}^\infty \ov{x}\,f(x)\ov{g(x)}\,dx$.  Here $x$ is real
   so $\ov{x}=x$ so $\ip{\phi(f),g}=\ip{f,\phi(g)}$, which means that
   $\phi$ is self-adjoint. 
  \item[(b)] If $\phi(f)=\lm\,f$ then $x,f(x)=\lm\,f(x)$ for all
   $x\in\R$, so $(x-\lm)\,f(x)=0$ for all $x\in\R$.  If $x\neq\lm$
   then we can divide by $x-\lm$ to see that $f(x)=0$.  If
   $\lm\not\in\R$ then this finishes the argument: $x$ can never be
   equal to $\lm$, so $f(x)=0$ for all $x\in\R$, so $f=0$.  If $\lm$
   is real then we need one extra step: we have seen that $f(x)=0$ for
   all $x\neq\lm$, but $f$ is continuous so it cannot jump away from
   zero at $x=\lm$, so $f(\lm)=0$ as well, so $f=0$.
 \end{itemize}
\end{solution}

\begin{exercise}
 Let $T$ be the matrix $\bpm 0&1\\1&0\epm$, and define
 $\gm\:M_2\R\to M_2\R$ by $\gm(A)=TA-AT$.
 \begin{itemize}
  \item[(a)] Give a basis for $M_2\R$, and find the matrix of
  $\gm$ with respect to that basis.
  \item[(b)] Find bases for the kernel and the image of $\gm$.
  Show that the image is the orthogonal complement of the kernel
  with respect to the usual inner product $\ip{X,Y}=\trace(XY^T)$
  on $M_2\R$.
  \item[(c)] Show that $\gm^4=4\gm^2$.
  \item[(d)] Find a basis of $M_2\R$ consisting of eigenvectors
  for $\gm$.  (Note here that an eigenvector for $\gm$ is a
  \emph{matrix} $A$ such that $\gm(A)=TA-AT=\lm A$ for some
  $\lm$.)
 \end{itemize}
\end{exercise}
\begin{solution}
 We first note that
 \[ \gm\bpm a&b\\ c&d\epm =
    \bpm 0&1\\ 1&0\epm \bpm a&b\\ c&d\epm -
    \bpm a&b\\ c&d\epm \bpm 0&1\\ 1&0\epm =
    \bpm c&d\\ a&b\epm - \bpm b&a\\ d&c\epm =
    \bpm c-b & d-a \\ a-d & b-c \epm.
 \]
 \begin{itemize}
  \item[(a)] The standard basis for $M_2\R$ consists of the
   matrices
   \[ E_1 = \bpm 1&0\\0&0 \epm \hspace{4em}
      E_2 = \bpm 0&1\\0&0 \epm \hspace{4em}
      E_3 = \bpm 0&0\\1&0 \epm \hspace{4em}
      E_4 = \bpm 0&0\\0&1 \epm
   \]
   We have
   \begin{align*}
    \gm(E_1) &= \bpm 0&-1\\1&0 \epm = -E_2+E_3 \\
    \gm(E_2) &= \bpm -1&0\\0&1 \epm = -E_1+E_4 \\
    \gm(E_3) &= \bpm 1&0\\0&-1 \epm =  E_1-E_4 \\
    \gm(E_4) &= \bpm 0&1\\-1&0 \epm =  E_2-E_3
   \end{align*}
   so the matrix of $\gm$ with respect to $E_1,E_2,E_3,E_4$ is
   \[ \bpm  0 & -1 &  1 &  0 \\
           -1 &  0 &  0 &  1 \\
            1 &  0 &  0 & -1 \\
            0 &  1 & -1 &  0 \epm
   \]
  \item[(b)] From our formulae for $\gm(E_i)$, we see that the
   matrices $U=E_1-E_4=\bpm 1&0\\0&-1\epm$ and
   $V=E_2-E_3=\bpm 0&1\\-1&0\epm$ form a basis for $\img(\gm)$, and
   the matrices $I=E_1+E_4$ and $T=E_2+E_3$ form a basis for
   $\ker(\gm)$.  The matrices $E_1,\dotsc,E_4$ are orthonormal with
   respect to the usual inner product, so we have
   \begin{align*}
    \ip{U,I} &= \ip{E_1-E_4,E_1+E_4} = 1.1 + (-1).1 = 0 \\
    \ip{U,T} &= \ip{E_1-E_4,E_2+E_3} = 0 \\
    \ip{V,I} &= \ip{E_2-E_3,E_1+E_4} = 0 \\
    \ip{V,T} &= \ip{E_2-E_3,E_2+E_3} = 1.1 + (-1).1 = 0.
   \end{align*}
   This shows that $\ker(\gm)=\spn\{I,T\}$ is orthogonal to
   $\img(\gm)=\spn\{U,V\}$.  As the dimensions of these two
   subspaces add up to the dimension of the whole space, the
   subspaces must be orthogonal complements of each other.
  \item[(c)] One checks that $\gm(U)=-2V$ and $\gm(V)=-2U$.  It
   follows that $\gm^2(U)=4U$, and so
   $\gm^4(U)=\gm^2(4U)=16U=4\gm^2(U)$.  Similarly, we have
   $\gm^4(V)=16V=4\gm^2(V)$.  As $\gm(I)=\gm(T)=0$, we also have
   $\gm^4(I)=0=4\gm^2(I)$ and $\gm^4(T)=0=4\gm^2(T)$.  Thus $gm^4$
   and $4\gm^2$ have the same effect on $U,V,I$ and $T$, which span
   $M_2\R$, so $\gm^4=4\gm^2$.  Alternatively, we can just square
   the matrix in part~(a) twice to see that
   \[
    \gm^2\sim
     \bpm 2 &  0 &  0 & -2 \\
          0 &  2 & -2 &  0 \\
          0 & -2 &  2 &  0 \\
         -2 &  0 &  0 &  2 \epm
    \hspace{4em}
    \gm^4\sim
     \bpm 8 &  0 &  0 & -8 \\
          0 &  8 & -8 &  0 \\
          0 & -8 &  8 &  0 \\
         -8 &  0 &  0 &  8 \epm
   \]
   which again shows that $\gm^4=4\gm^2$.
  \item[(d)] As $T$ and $I$ lie in the kernel of $\gm$, they are
  eigenvectors with eigenvalue $0$.  As $\gm(U)=-2V$ and
  $\gm(V)=-2U$ we see that $\gm(U+V)=-2(U+V)$ and
  $\gm(U-V)=2(U-V)$.  It follows that $\{I,T,U-V,U+V\}$ is a basis
  consisting of eigenvectors.
 \end{itemize}
\end{solution}

\end{document}



%%% Local Variables:
%%% compile-command: "do_both 11"
%%% End:
