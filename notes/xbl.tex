\documentclass{amsart}
\usepackage{a4wide}
\usepackage{xr}

\externaldocument{linear}

\newcommand{\CV}        {{\mathcal{V}}}
\newcommand{\R}         {{\mathbb{R}}}
\newcommand{\img}{\operatorname{image}}
\renewcommand{\:}{\colon}

\begin{document}

\begin{center}\Large
 PMA220 Vector spaces and Fourier theory\\
 Guidance on the final exam
\end{center}
\vspace{4ex}

\section*{Definitions}

You may be asked to reproduce any of the following
definitions:
\begin{enumerate}
 \item Linear map (Definition~{3.1})
 \item Isomorphism (Definition~{3.18})
 \item Subspace (Definition~{4.1})
 \item $V+W$ (Definition~{4.13})
 \item Kernel, image (Definition~{4.19})
 \item Direct sum (Definition~{4.26})
 \item Linear relation, (in)dependence (Definition~{5.1})
%\item  Wronskian (Definition~{5.8})
 \item Span (Definition~{5.13})
 \item Finite-dimensional (Definition~{5.21})
 \item Basis (Definition~{5.24})
%\item  The matrix of a linear map
 \item Eigenvalues and eigenvectors (Definition~{9.1})
 \item Inner product (Definition~{10.1})
%\item  Angles (Definition~{11.5})
 \item Orthogonal complement (Definition~{12.1})
 \item Orthogonal and orthonormal sequences
  (Definition~{12.3})
 \item Hermitian form (Definition~{13.1})
 \item Adjoint matrix (Definition~{13.3})
 \item Adjoint map (Definition~{14.1})
 \item Continuous periodic functions (Definition~{15.1})
 \item Trigonometric polynomials (Definition~{15.2})
 \item Self-adjoint operator (Definition~{16.1})
\end{enumerate}

\section*{Results}
 
You may be asked to prove any of the following results.  You
may also be asked to prove simplified versions or special
cases, which should be easier provided that you actually
understand the proofs.
\begin{enumerate}
 \item $V\cap W$ and $V+W$ are subspaces (Proposition~{4.16})
 \item $\ker(\phi)\leq U$, and $\img(\phi)\leq V$
  (Proposition~{4.21})
 \item $\phi$ is injective iff $\ker(\phi)=\{0\}$, and
  $\phi$ is surjective iff $\img(\phi)=V$ (Proposition~{4.21})
%\item  $\R[x]$ is not finite-dimensional
 \item Any $\phi\:\R^n\to V$ has $\phi=\mu_\CV$ for some
  $\CV$ (Proposition~{6.3})
 \item Adapted bases for a pair of subspaces
  (Proposition~{8.14})
 \item Adapted bases for a linear map (Theorem~{8.17})
 \item Rank-nullity formula (Corollary~{8.19})
 \item The Cauchy-Schwartz inequality (Theorem~{11.1})
%\item  $W\cap W^\perp=0$ (Lemma~{12.2})
 \item Pythagoras equation (Lemma~{12.7})
 \item Any strictly orthogonal sequence is linearly
  independent (Lemma~{12.8})
%\item  Projector formula (Proposition~{12.9})
 \item Parseval's inequality (Corollary~{12.11})
 \item Projections are closest (Proposition~{12.12})
%\item  Self-adjoint operators have real eigenvalues and orthogonal eigenspaces.
\end{enumerate}
\vspace{4ex}
 
\section*{Calculations}

You may be asked to carry out any of the following calculations. 
\begin{enumerate}
 \item Check whether a map is linear
 \item Compute values for a given linear map
 \item Check whether a set is a subspace
 \item Find a basis (or just the dimension) for the kernel or image of
  a map
 \item Find a basis (or just the dimension) for the span of a list
 \item Find a basis (or just the dimension) for the sum or
  intersection of two subspaces
 \item Check whether a map is injective, surjective or an isomorphism
 \item Numerical criteria for injectivity, surjectivity, bijectivity
 \item Find a preimage
 \item Find a nontrivial element of the kernel
 \item Compute values of $\mu_{\mathcal{V}}$
 \item Check whether a list is independent, or spans, or is a basis
 \item Numerical criteria for independence, spanning and basishood
 \item Find a preimage under $\mu_{\mathcal{V}}$
 \item Find a linear relation
 \item Find and use the Wronskian
 \item Find the matrix of a linear map with respect to specified bases
 \item Find adapted bases for a linear map
 \item Numerical deductions from rank-nullity and sum dimension
  formulae
 \item Verify eigenvectors and eigenvalues
 \item Find eigenvectors and eigenvalues
 \item Calculate some inner products
 \item Given a linear form and an inner product, find the representing vector
 \item Work with concrete cases of the Cauchy-Schwartz inequality
 \item Carry out the Gram-Schmidt procedure
\end{enumerate}


\end{document}
